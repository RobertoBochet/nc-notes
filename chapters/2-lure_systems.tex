\chapter{Lur'e systems}

\section{Definition}

\begin{figure}[htb]
    \centering
    \resizebox{0.7\textwidth}{!}{
        \begin{tikzpicture}
            \node [sum] (sum_e) {};

            \node [block, right=1cm of sum_e] (controller) {$\varphi(\cdot)$};
            \node [block, right=1cm of controller] (system) {$G(s)$};

            \node [output, right=1cm of system] (oy) {};

            \draw [->] (sum_e) -- node {$e$} (controller);
            \draw [->] (controller) -- node {$u$} (system);
            \draw [->] (system) -- node [pos=0.8] {$y$} node [node, name=y, pos=0.5] {} (oy);

            \draw [->] (y) -- ++(0,-1.5) -| node [pos=0.95] {$-$} (sum_e);
        \end{tikzpicture}
    }
    \caption{Autonomous Lur'e system}
    \label{fig:lure-system}
\end{figure}

\textbf{Autonomous Lur'e system} is composed by a \textbf{linear system} ($G(s)$) \textbf{controllable} and \textbf{observable} fed by a \textbf{nonlinear function} ($\varphi(e)$) on the feedback error.

\[
    \begin{cases}
        \dx = A x + B \varphi(-Cx) \\
        y = C x
    \end{cases}
\]

The non linear function is bounded into a sector ($\varphi(e) \in \Phi_{[k_1,k_2]}$) defined as

\[
    \Phi_{[k_1,k_2]} = \{\phi(\cdot) : (k_2 e -u)(u - k_1 e) \geq 0, u=\phi(e), \forall e \in \Real \}
\]

For this kind of system $\bar{x} = 0$ is an equilibrium.

\subsection{Meaning}

\begin{figure}[htb]
    \centering
    \resizebox{.9\textwidth}{!}{
    \begin{tikzpicture}
        \node [input] (iy_d) {};

        \node [sum, right=1cm of iy_d] (sum_e) {};
        \node [block, right=0.6cm of sum_e] (controller) {$R(s)$};
        \node [sum, right=0.6cm of controller] (sum_u) {};
        \node [block, right=0.6cm of sum_u] (nonlinearity) {$\Psi(\cdot)$};
        \node [block, right=1cm of nonlinearity] (system) {$P(s)$};
        \node [sum, right=0.6cm of system] (sum_d) {};

        \node [output, right=1cm of sum_d] (oy) {};

        \node [input, above=1cm of sum_u] (iu_n) {};
        \node [input, above=1cm of sum_d] (id_n) {};

        \node [block, below=1cm of nonlinearity] (feedback) {$T(s)$};

        \draw [->] (iy_d) -- node{$y^0$} (sum_e);
        \draw [->] (sum_e) -- node {$e$} (controller);
        \draw [->] (controller) -- node {$w$} (sum_u);
        \draw [->] (sum_u) -- node {$u$} (nonlinearity);
        \draw [->] (nonlinearity) -- node {$m$} (system);
        \draw [->] (system) -- (sum_d);

        \draw [->] (sum_d) -- node [pos=0.8] {$\tilde y$} node [node, name=y, pos=0.5] {} (oy);

        \draw [->] (y) |- (feedback);
        \draw [->] (feedback) -| node[pos=0.07] {$y$} node [pos=0.95] {$-$} (sum_e);

        \draw [->] (iu_n) -- node[pos=0.2] {$u^0$} (sum_u);
        \draw [->] (id_n) -- node[pos=0.2] {$d^0$} (sum_d);

    \end{tikzpicture}
    }
    \caption{A generic feedback system with a nonlinearity}
    \label{fig:generic-feedback-system}
\end{figure}

A general feedback system (e.g \cref{fig:generic-feedback-system}) containing a nonlinear function where input are constant can be traced to an \textbf{autonomous Lur'e system}.

Chosen an equilibrium (associated to a constant inputs) and the whole system can be studied in the neighbourhood of it applying a coordinate shifting (i.e. $\Delta x(t) = x(t) - \bar x$)

From the definition of constant inputs we get $\Delta y^0 = \Delta u^0 = \Delta d^0 = 0$, then we get a system like \cref{fig:generic-feedback-system-equilibrium}, easily to retract back to an \textbf{Autonomous Lur'e system}

\begin{figure}[htb]
    \centering
    \resizebox{.8\textwidth}{!}{
        \begin{tikzpicture}
            \node [sum] (sum_e) {};
            \node [block, right=0.6cm of sum_e] (controller) {$R(s)$};
            \node [block, right=2cm of controller] (nonlinearity) {$\varphi(\cdot)$};
            \node [block, right=1cm of nonlinearity] (system) {$P(s)$};

            \node [output, right=1cm of system] (oy) {};

            \node [block, below=1cm of nonlinearity] (feedback) {$T(s)$};

            \draw [->] (sum_e) -- node {$\Delta e$} (controller);
            \draw [->] (controller) -- node {$\Delta w = \Delta u$} (nonlinearity);
            \draw [->] (nonlinearity) -- node {$\Delta m$} (system);
            \draw [->] (system) -- node [pos=0.8] {$\Delta \tilde y$} node [node, name=y, pos=0.5] {} (oy);

            \draw [->] (y) |- (feedback);
            \draw [->] (feedback) -| node[pos=0.07] {$\Delta y$} node [pos=0.95] {$-$} (sum_e);


        \end{tikzpicture}
    }
    \caption{A generic feedback system with a nonlinearity considered near the equilibrium}
    \label{fig:generic-feedback-system-equilibrium}
\end{figure}

\section{Properties}

\begin{theorem}[Nyquist criterion]\label{thm:nyquist}
A degenerated \textbf{autonomous Lur'e system} with $\varphi(\cdot) = k$ (linearly in the error) is asymptotically stable if and only if the Nyquist plot of the linear part encircles (anti-clockwise) the point of the complex plane $-\frac{1}{k}$ as many times as the amount the unstable poles of it
\end{theorem}

\begin{theorem}\label{thm:lure-nyquist}
    \[
        I(k_1,k_2) = \{\alpha \in \Real : \alpha = -\frac{1}{k}, k \in [k_1,k_2]\}
    \]
    If \textbf{autonomous Lur'e system} is absolutely stable in the sector $[k_1,k_2]$, then the Nyquist plot of the linear part encircles (anti-clockwise) $I(k_1,k_2)$ as many times as the amount the unstable poles of it
\end{theorem}

\begin{nb}\cref{thm:lure-nyquist} defined a necessary condition for stability but not sufficient\end{nb}

\begin{conjecture}[Aizerman conjecture, 1949]\label{cjt:aizerman}
    The \cref{thm:lure-nyquist} is also sufficient for the stability of the \textbf{autonomous Lur'e system}
\end{conjecture}

\begin{nb}
    it is possible find a counterexample for the \cref{cjt:aizerman}
\end{nb}

\subsection{Popov criterion}

\begin{theorem}[Popov criterion, 1962]\label{thm:popov-criterion}
    An \textbf{autonomous Lur'e system} is absolutely stable in sector $[0,k]$ if the linear part is asymptotically stable and exists a real number $q$ such that
    \[
        \Re [(1+\j \omega q) G(\j \omega)] > - \frac{1}{k}, \forall \omega \geq 0
    \]
    is satisfied
\end{theorem}

\begin{nb}\cref{thm:popov-criterion} defined a sufficient condition for stability\end{nb}

\subsubsection{Special case $q=0$}

With $q=0$ the condition is reduced to

\[
    \Re [G(\j \omega)] > - \frac{1}{k}, \forall \omega \geq 0
\]

so the \cref{thm:popov-criterion} requires that the $G(\j\omega)$ polar plot is contained in the right-half-plane defined by the vertical line passing to the real point $-\frac{1}{k}$

\subsubsection{Case $q \neq 0$}

\begin{corollary}
    Defining
    \[
        G^*(\j\omega) = \Re[G(\j\omega)] + \j\omega \Im[G(\j\omega)]
    \]
    the \cref{thm:popov-criterion} requires that exists a straight line passing through the real point $-\frac{1}{k}$ such that the $G^*(\j\omega)$ polar plot is contained in the right-half-plane defined by it
\end{corollary}

\subsubsection{Popov criterion and Aizenman conjecture}

Let us denote with $K_P$ the largest $K$ value such that the absolute stability of an \textbf{autonomous Lur'e system} in the sector $[0,K]$ is guaranteed via \textbf{Popov criterion};
and with $K_N$ the largest $K$ such that the degenerated \textbf{autonomous Lur'e system} ($\varphi(\cdot)$ linearly in error) is asymptotically stable via \cref{thm:nyquist}(\textbf{Nyquist criterion})

\begin{nb} for an \textbf{autonomous Lur'e system} $K_P \leq K_N$\end{nb}

\begin{nb}if $K_P = K_N$ then, the system satisfies the \textbf{Aizenman Conjecture} \end{nb}

\begin{nb}$K_P < K_N$ does not imply that the system does not satisfy the \textbf{Aizenman Conjecture}\end{nb}