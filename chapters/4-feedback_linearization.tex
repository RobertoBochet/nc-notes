\usepackage{amsmath}\chapter{Feedback linearization}

\section{Definition}

Given a nonlinear system in the form

\[
    \begin{cases}
        \dx =\vect a(\x) + \vect b(\x) u \\
        y = \vect c(\x)
    \end{cases}
\]

design a static state feedback control law $u = k(\x,v)$ such that the associated feedback system is linear.

\begin{nb} the system is linear w.r.t. input \end{nb}

\begin{example}
    Model of a centrifuge

    \[
        J \ddot\theta = -k \dot \theta^2 \sgn(\dot \theta) + u
    \]

    setting $x = \dot\theta$ the system can be written

    \[
    \begin{cases}
        \dot x = - \frac{k}{J} x^2 \sgn(x) + \frac{1}{J} u \\
        y = x
    \end{cases}
    \]

    if we set $v = - \frac{k}{J}x^2 \sgn(x) + \frac{1}{J}u$ the system become

    \[
    \begin{cases}
        \dot x = v \\
        y = x
    \end{cases}
    \]

    the feedback system seen from outside seems linear, and a controller for linear system can be made using the input $v$ and the output $y$.
\end{example}

\subsection{Relative degree}

Before to go on, let us introduce the concept of \textbf{relative degree} of a system.

Given a system with output $y=\vect c(\x)$ its relative degree is defined as the minimum amount time derivation of $y$ such that it is directly dependent by input $u$.

\[
    r = \min \left\{ \gamma : \pd{y^{(\gamma)}}{u} \neq 0 \right\}
\]

The concept of relative degree can be formulated also exploiting the \textbf{Lie derivative}\footnote{Defined as $\Lie{f} h(\x) = \pd{h}{\x} f(\x) = \sum_{i=1}^{n} f_i(\x) \pd{h(\x)}{x_i}$}

\[
    \dot y =
    %\frac{d}{dt} \vect c(\x) =
    \pd{\vect c}{\x} \dx =
    \pd{\vect c}{\x} \left(\vect a(\x) + \vect b(\x) u\right) =
    \Lie{\vect a} \vect c(\x) + u \Lie{\vect b} \vect c(\x)
\]

if $\Lie{\vect b} \vect c(\x) \neq 0$ so the relative degree of the system is equal to $1$ else it is greater than $1$ or not well-defined;
in this case we can iterate the derivation of the output in order to find the relative degree.

Since $\Lie{\vect b} \vect c(\x) = 0$ then $\dot y = \Lie{\vect a} \vect c(\x)$ and

\[
    \ddot y =
    %\frac{d}{dt} \left( \Lie{\vect a} \vect c(\x) \right) =
    \pd{\left( \Lie{\vect a} \vect c(\x) \right)}{x} \dx =
    \pd{\left( \Lie{\vect a} \vect c(\x) \right)}{x} \left(\vect a(\x) + \vect b(\x) u\right) =
    \Lie{\vect a}^2 \vect c(\x) + u \Lie{\vect b} \Lie{\vect a} \vect c(\x)
\]

now if $\Lie{\vect b} \Lie{\vect a} \vect c(\x) \neq 0$ then the relative degree is $2$ else we can go on with the iterations.

The procedure can be generalized

\begin{equation}\label{eqn:output-via-lie}
    y^{(k)} = \Lie{\vect a}^k \vect c(\x) + u \Lie{\vect b} \Lie{\vect a}^{k-1} \vect c(\x) \qquad \Longleftarrow \Lie{\vect b} \Lie{\vect a}^{h-1} \vect c(\x) = 0, h < k
\end{equation}

so the relative degree can be defined as

\[
    r = \min \left\{ \gamma : \Lie{\vect b} \Lie{\vect a}^{\gamma-1} \vect c(\x) \neq 0 \right\}
\]

\subsection{Normal canonical form}

Given a nonlinear, affine, time-invariant, SISO system that can be rewritten in the \textbf{normal canonical form}, in particular a system of degree $n$ and relative degree $r$ with input $u$ and state $\x$ can be rewritten in the form

\[
    \begin{cases}
        \dot{\tilde x}_1 = \tilde x_2 \\
        \dot{\tilde x}_2 = \tilde x_3 \\
        \vdots \\
        \dot{\tilde x}_r = \tilde\lambda(\vect{\tilde x}) + \tilde\mu (\vect{\tilde x}) u \\
        \left.\begin{matrix}\dot{\tilde x}_{r+1} = \tilde\eta_1(\vect{\tilde x}) \\
        \vdots \\
        \dot{\tilde x}_{n} = \tilde\eta_{n-r}(\vect{\tilde x}) \\
        \end{matrix} \qquad \right. \\
        y = \tilde x_1
    \end{cases}
\]

where the state $\vect{\tilde x}$ is a transformation of the state $\x$ of the original system under the transformation $\vect{\tilde x} = \varphi(\vect x)$.

\subsubsection{Linearization of normal canonical form}

If we impose

\[
    u = \frac{1}{\tilde\mu (\vect{\tilde x})} \left( v - \tilde\lambda(\vect{\tilde x}) \right)
\]

we get a linear I/O map

\[
    \begin{cases}
        \dot{\tilde x}_1 = \tilde x_2 \\
        \dot{\tilde x}_2 = \tilde x_3 \\
        \vdots \\
        \dot{\tilde x}_r = v \\
        \left.\begin{matrix}\dot{\tilde x}_{r+1} = \tilde\eta_1(\vect{\tilde x}) \\
        \vdots \\
        \dot{\tilde x}_{n} = \tilde\eta_{n-r}(\vect{\tilde x}) \\
        \end{matrix} \qquad \right\} \text{zero dynamics} \\
        y = \tilde x_1
    \end{cases}
\]

\begin{nb}if $r=n$ the "zero dynamics" part disappears, and we get a fully linearized system\end{nb}

\section{Linearization}

So the goal given a generic nonlinear, affine, time-invariant, SISO system is to define a transformation $\vect{\tilde x} = \varphi(\vect x)$ which defines an equivalent system in the \textbf{normal canonical form}.

From the definition of the \textbf{relative degree}(\cref{eqn:output-via-lie}) we have

\[
    y^{(k-1)}=\Lie{a}^{k-1} c(\x),\qquad k = 1,2,\dots, r-1
\]

so we can impose $\tilde x_k =\varphi_k(\x) = \Lie{a}^{k-1} c(\x)$, then

\begin{multline*}
    \dot{\tilde x}_k =
    \pd{\varphi_k}{\x} \dx =
    \pd{\left( \Lie{a}^{k-1} c(\x) \right)}{\x} \left( a(\x) + b(\x) u \right) \\ =
    \Lie{a}^k c(\x) + \cancelto{0}{u \Lie{b}\Lie{a}^{k-1} c(\x)} =
    \Lie{a}^k c(\x) =
    \varphi_{k+1}(\x) =
    \tilde x_{k+1}
\end{multline*}

so
\[
    \dot{\tilde x}_k = \tilde x_{k+1},\qquad k = 1,2,\dots, r-1
\]

instead the differential of $\dot{\tilde x}_r$ can be written as

\begin{equation}\label{eqn:newx-r}
    \dot{\tilde x}_r = \Lie{a}^r c(\x) + u \Lie{b}\Lie{a}^{r-1} c(\x)
\end{equation}

\begin{nb}the transformation $\tilde x_k =\varphi_k(\x)$ imposes $y̛^{(k-1)} = \tilde x_k,\quad k = 1,2,\dots,r$\end{nb}

\begin{theorem}[input-output state feedback linearization]
If a nonlinear, affine, time-invariant, SISO system has \textbf{relative degree} $r$ then, one can get a linear I/O map via state feedback with the control law
    \[
        u = \frac{1}{\Lie{\vect b} \Lie{\vect a}^{r-1} \vect c(\x)} \left( v - \Lie{\vect a}^{r} \vect c(\x) \right)
    \]
\end{theorem}

\begin{proof}
    Using \cref{eqn:newx-r} we can define the equation for the control law as

    \[
        v = \Lie{\vect a}^{r} \vect c(\x) + u \Lie{\vect b} \Lie{\vect a}^{r-1} \vect c(\x)
    \]

    and if we merge the two equations we get

    \[
        \dot{\tilde x}_r = v \quad \implies \quad y^{(r)} = v
    \]
\end{proof}

\subsection{Full feedback linearization}

\begin{theorem}\label{thm:full-feedback-linearization}
A \textbf{full feedback linearization} of a nonlinear, affine, time-invariant, SISO system is possible if and only if a regular function $c(\cdot)$ can be found such that system
\[
    \begin{cases}
        \dx = \vect a(\x) + \vect b(\x) u \\
        y = \vect c(\x)
    \end{cases}
\]
has a \textbf{relative degree} $r$ equal to the order $n$ of the linear system.
\end{theorem}

so we get the linearized system

\[
    \begin{cases}
        \dxe = \tilde A \xe + \tilde B v \\
        y = \tilde C \xe
    \end{cases} \quad
    \tilde A = \begin{bmatrix}
            0 & 1 & 0 & \dots & 0 \\
            0 & 0 & 1 & \dots & 0 \\
            \dots & \dots & \dots & \dots & \dots \\
            0 & 0 & 0 & \dots & 1 \\
            0 & 0 & 0 & \dots & 0 \\
    \end{bmatrix},
    \tilde B = \begin{bmatrix}
                   0 \\ 0 \\ \dots \\ 0 \\ 1
    \end{bmatrix},
    \tilde C = \begin{bmatrix}
                   1 \\ 0 \\ \dots \\ 0 \\ 0
    \end{bmatrix}^{\trans}
\]
