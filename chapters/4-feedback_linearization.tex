\chapter{Feedback linearization}

\section{Definition}

Given a nonlinear system in the form

\[
    \begin{cases}
        \dot x = a(x) + b(x) u \\
        y = c(x)
    \end{cases}
\]

design a static state feedback control law $u = k(x,v)$ such that the associated feedback system is linear.

\begin{nb} the system is linear w.r.t. input\end{nb}

\begin{example}
    Model of a centrifuge

    \[
        J \ddot\tetha = -k \dot \theta^2 \sgn(\dot \theta) + u
    \]

    setting $x = \dot\theta$ the system can be written

    \begin{cases}
        \dot x = - \frac{k}{J} x^2 \sgn(x) + \frac{1}{J} u \\
        y = x
    \end{cases}

    if we set $u = - \frac{k}{J}x^2 \sgn(x) + \frac{1}{J}u$ the system become

    \begin{cases}
        \dot x = v \\
        y = x
    \end{cases}

    the feedback system seen from outside seems linear, and a controller for linear system can be made using the input $v$ and the output $y$.
\end{example}