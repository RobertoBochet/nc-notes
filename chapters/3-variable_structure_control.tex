\chapter{Variable structure control}

\section{Definition}

\begin{figure}[htb]
    \centering
    \resizebox{0.7\textwidth}{!}{
        \begin{tikzpicture}
            \node [sum] (sum_e) {};

            \node [block, right=1cm of sum_e] (controller) {$\psi(\cdot)$};
            \node [block, above=1cm of controller] (switching_law) {$s(\cdot) > 0$};
            \node [block, right=1cm of controller] (system) {$G(s)$};

            \node [output, right=1cm of system] (oy) {};

            \draw [->] (sum_e) -- node {$e$} (controller);
            \draw [->] (switching_law) -- (controller);
            \draw [->] (controller) -- node {$u$} (system);
            \draw [->] (system) |- node [pos=0.8] {$x$} (switching_law);
            \draw [->] (system) -- node [pos=0.8] {$y$} node [node, name=y, pos=0.5] {} (oy);

            \draw [->] (y) -- ++(0,-1.5) -| node [pos=0.95] {$-$} (sum_e);
        \end{tikzpicture}
    }
    \caption{Example of VSC}
    \label{fig:vsc-example}
\end{figure}

In \cref{fig:vsc-example} is shown an example of the structure of a typical \textbf{VSC}, in particular we have a piecewise control law, based on the switching function $s(x) > 0$.

\[
    \psi(e) = \begin{cases}
                  \varphi_1(e), \qquad s(x) > 0 \\
                  \varphi_2(e), \qquad s(x) < 0
              \end{cases}
\]

given this control law we get two different overall systems, one based on the control law for $s(x)>0$ and one for $s(x)<0$ (\cref{fig:vcs-two-systems}).

\begin{figure}[htb]
    \centering
    \begin{subfigure}[b]{0.45\textwidth}
        \resizebox{\textwidth}{!}{
            \begin{tikzpicture}
                \node [sum] (sum_e) {};

                \node [block, right=1cm of sum_e] (controller) {$\varphi_1(\cdot)$};
                \node [block, right=1cm of controller] (system) {$G(s)$};

                \node [output, right=1cm of system] (oy) {};

                \draw [->] (sum_e) -- node {$e$} (controller);
                \draw [->] (controller) -- node {$u$} (system);
                \draw [->] (system) -- node [pos=0.8] {$y$} node [node, name=y, pos=0.5] {} (oy);

                \draw [->] (y) -- ++(0,-1.5) -| node [pos=0.95] {$-$} (sum_e);
            \end{tikzpicture}
        }
        \caption{$s(x)>0$}
        \label{fig:vcs-two-systems-a}
    \end{subfigure}
    \hfill
    \begin{subfigure}[b]{0.45\textwidth}
        \resizebox{\textwidth}{!}{
            \begin{tikzpicture}
                \node [sum] (sum_e) {};

                \node [block, right=1cm of sum_e] (controller) {$\varphi_2(\cdot)$};
                \node [block, right=1cm of controller] (system) {$G(s)$};

                \node [output, right=1cm of system] (oy) {};

                \draw [->] (sum_e) -- node {$e$} (controller);
                \draw [->] (controller) -- node {$u$} (system);
                \draw [->] (system) -- node [pos=0.8] {$y$} node [node, name=y, pos=0.5] {} (oy);

                \draw [->] (y) -- ++(0,-1.5) -| node [pos=0.95] {$-$} (sum_e);
            \end{tikzpicture}
        }
        \caption{$s(x)<0$}
        \label{fig:vcs-two-systems-b}
    \end{subfigure}
    \caption{The two systems under the switching law}
    \label{fig:vcs-two-systems}
\end{figure}

\begin{nb}the stability of the two systems tells us nothing about the stability of the overall one\end{nb}

