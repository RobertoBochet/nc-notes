\documentclass[12pt]{report}

\usepackage[utf8]{inputenc}
\usepackage{amssymb}
\usepackage{amsmath}
\usepackage{svg}
\usepackage{tabularx}
\usepackage{soul}
\usepackage[pdfusetitle]{hyperref}
\usepackage{enumitem}
\usepackage{nicefrac}
\usepackage[capitalize,nameinlink,noabbrev]{cleveref}

\usepackage{geometry}
\geometry{
a4paper,
left=30mm,
right=30mm,
top=40mm,
bottom=30mm
}

\setcounter{tocdepth}{3}
\setcounter{secnumdepth}{3}

\usepackage{tikz}
\usetikzlibrary{circuits}
\usetikzlibrary{shapes,arrows}
\usetikzlibrary{positioning}

\tikzset{auto,>=latex', minimum size=0pt, node distance=1, font=\fontsize{11}{11}\selectfont}

\tikzstyle{block} = [draw, fill=white, rectangle, minimum height=3em, minimum width=4em, text centered, align=center]
\tikzstyle{integrator} = [draw, fill=white, rectangle, minimum height=3em, minimum width=2em, font={$\frac{1}{s}$}]
\tikzstyle{sum} = [draw, fill=white, circle, node distance=1cm]
\tikzstyle{input} = [coordinate]
\tikzstyle{output} = [coordinate]
\tikzstyle{pinstyle} = [pin edge={to-,thin,black}]
\tikzstyle{gain} = [draw, isosceles triangle, minimum height = 3em, isosceles triangle apex angle=60]
\tikzstyle{spy} = [coordinate, inner sep=0, outer sep=0, minimum size=0]
\tikzstyle{node} = [draw, circle, fill, minimum size=2pt, inner sep=0pt, outer sep=0pt, anchor=center]

%\usepackage{bodegraph}

%\usetikzlibrary{intersections}
%\usetikzlibrary{calc}

\setlength{\parindent}{0em}
\setlength{\parskip}{0.5em}

\newenvironment{conditions}
{\par\vspace{\abovedisplayskip}\noindent\centering\begin{tabular}{>{$}r<{$} @{${}:{}$} l}}
{\end{tabular}\par\vspace{\belowdisplayskip}}

\newenvironment{nb}{\par\vspace{\abovedisplayskip}\noindent\begin{em}n.b.\ \ignorespaces}{\end{em}}

\newcommand{\pd}[2]{\frac{\partial #1}{\partial #2}}


%\DeclareMathOperator{\Re}{Re}
%\DeclareMathOperator{\Im}{Im}
\DeclareMathOperator{\rad}{rad}
\DeclareMathOperator{\sgn}{sgn}
\DeclareMathOperator{\diag}{diag}

\newcommand{\matr}[1]{\boldsymbol{#1}}
\newcommand{\vect}[1]{\boldsymbol{#1}}

\newcommand{\trans}{\mathsf{T}}
\newcommand{\inv}{{-1}}
\newcommand{\psinv}{\dag}
\newcommand\norm[1]{\left\lVert#1\right\rVert}

\newcommand{\Real}{\mathbb{R}}
\newcommand{\Complex}{\mathbb{C}}

\newcommand{\I}{\matr{I}}
\newcommand{\0}{\vect{0}}

\newcommand{\J}{\matr{J}}
\newcommand{\dJ}{\dot{\J}}

\newcommand{\estimate}[1]{\hat{#1}}
\newcommand{\error}[1]{\tilde{#1}}
\newcommand{\reference}[1]{\bar{#1}}

\newcommand{\B}{\matr{B}}
\newcommand{\dB}{\dot\B}
\newcommand{\Bs}{\hat\B}
\newcommand{\Be}{\error\B}

\newcommand{\C}{\matr{C}}
\newcommand{\Cs}{\estimate\C}
\newcommand{\Ce}{\error\C}

\newcommand{\g}{\vect{g}}
\newcommand{\gs}{\estimate\g}

\newcommand{\n}{\vect{n}}
\newcommand{\ns}{\estimate\n}
\newcommand{\ner}{\error\n} % #r becuase ne already exists

\newcommand{\K}{\matr{K}}

\newcommand{\y}{\vect{y}}
\newcommand{\dy}{\dot{\y}}
\newcommand{\ddy}{\ddot{\y}}

\newcommand{\V}{V}
\newcommand{\dV}{\dot{\V}}

\newcommand{\q}{\vect{q}}
\newcommand{\dq}{\dot{\q}}
\newcommand{\ddq}{\ddot{\q}}
\newcommand{\qd}{\bar{\q}}
\newcommand{\dqd}{\dot{\reference{\q}}}
\newcommand{\ddqd}{\ddot{\reference{\q}}}
\newcommand{\qe}{\error{\q}}
\newcommand{\dqe}{\dot{\error{\q}}}
\newcommand{\ddqe}{\ddot{\error{\q}}}
\newcommand{\qs}{\estimate{\q}}
\newcommand{\dqs}{\dot{\estimate{\q}}}
\newcommand{\ddqs}{\ddot{\estimate{\q}}}

\newcommand{\x}{x}
\newcommand{\dx}{\dot{\x}}
\newcommand{\ddx}{\ddot{\x}}
\newcommand{\xd}{\reference{\x}}
\newcommand{\dxd}{\dot{\reference{\x}}}
\newcommand{\ddxd}{\ddot{\reference{\x}}}
\newcommand{\xe}{\error{\x}}
\newcommand{\dxe}{\dot{\error{\x}}}
\newcommand{\ddxe}{\ddot{\error{\x}}}
\newcommand{\xs}{\estimate{\x}}
\newcommand{\dxs}{\dot{\estimate{\x}}}
\newcommand{\ddxs}{\ddot{\estimate{\x}}}

\newtheorem{theorem}{Theorem}[chapter]
\newtheorem{corollary}{Corollary}[theorem]
\newtheorem{lemma}[theorem]{Lemma}
\newtheorem{conjecture}{Conjecture}[chapter]

\newcommand{\warning}[1]{\textcolor{red}{WARNING #1}}

\usepackage{imakeidx}
\makeindex[intoc]

\usepackage[backend=biber]{biblatex}
\addbibresource{references.bib}

\title{Nonlinear control\\Notes\thanks{the whole document is based on \cite{cir-slides}}}
\author{Roberto Bochet}

\begin{document}

\maketitle

\printbibliography

\tableofcontents

\chapter{Passive systems}

\section{Definition}

\begin{gather*}
    \begin{cases}
        \dx (t) = f(\x,u) \\
        y(t) = g(\x,u)
    \end{cases} \\
    f(\vect 0,0) = \vect 0, \quad g(\vect 0,0)=0
\end{gather*}

A dynamical system is called \textbf{passive} if there exists a \textbf{Lyapunov function} $\V(\x)$ called \textbf{storage function}, such that

\[
    uy \geq \pd{V}{\x}(\x) f(\x,u) + \epsilon u^2 + \delta y^2 + \rho \psi(\x)
\]

with

\[
    \epsilon \geq 0, \delta \geq 0, \rho \geq 0, \psi(\x) > 0
\]

Notice that the Lyapunov function can be written as

\[
     \pd{\V}{\x} f(\x,0) = \dV(\x)
\]

\subsection{Conservative system}

A passive system is called \textbf{conservative} if $\epsilon = \delta = \rho = 0$, such that

\[
    uy = \pd{V}{\x}(\x) f(\x,u)
\]

\subsection{Strictly passive}

A passive system is called strictly passive if any of $\epsilon, \delta, \rho$ is greater that $0$.
In particular

Strictly passive w.r.t the input if $\epsilon > 0$ \\
Strictly passive w.r.t the output if $\delta > 0$ \\
Strictly passive w.r.t the state if $\rho > 0$

\subsection{Zero-state observable}

A system is \textbf{zero-state observable} if $\x(t) = 0$ is the only \textbf{free evolution} (u=0) of the state compatible with identically zero output (y=0)

\section{Properties}

\begin{theorem}
    If a system is \textbf{passive} with positive definite storage function, then $\x=0$ is a \textbf{stable} equilibrium for the system with zero input
\end{theorem}

\begin{theorem}
    If a system is \textbf{strictly passive w.r.t the state} with positive definite storage function, then $\x=0$ is an \textbf{asymptotically stable} equilibrium for the system with zero input
\end{theorem}

\begin{theorem}
    If a system is \textbf{strictly passive w.r.t the output} and \textbf{zero-stable observable} with positive definite storage function, then $\x=0$ is an \textbf{asymptotically stable} equilibrium for the system with zero input
\end{theorem}

\begin{theorem}
    If the storage function associated to \textbf{asymptotically stable} equilibrium is radially unbounded then the equilibrium is \textbf{GAS}
\end{theorem}

\begin{theorem}
    If a system is \textbf{zero state observable} and \textbf{passive} with positive definite storage function and \textbf{radially unbounded}, then $\x=0$ is a \textbf{GAS equilibrium} of the output feedback system with control law
    \[
        u=-\phi(y),\qquad \phi(0)=0, \quad y\phi(y)>0, \forall y \neq 0
    \]
\end{theorem}
\chapter{Lur'e systems}

\section{Definition}

\begin{figure}[htb]
    \centering
    \resizebox{0.7\textwidth}{!}{
        \begin{tikzpicture}
            \node [sum] (sum_e) {};

            \node [block, right=1cm of sum_e] (controller) {$\varphi(\cdot)$};
            \node [block, right=1cm of controller] (system) {$G(s)$};

            \node [output, right=1cm of system] (oy) {};

            \draw [->] (sum_e) -- node {$e$} (controller);
            \draw [->] (controller) -- node {$u$} (system);
            \draw [->] (system) -- node [pos=0.8] {$y$} node [node, name=y, pos=0.5] {} (oy);

            \draw [->] (y) -- ++(0,-1.5) -| node [pos=0.95] {$-$} (sum_e);
        \end{tikzpicture}
    }
    \caption{Autonomous Lur'e system}
    \label{fig:lure-system}
\end{figure}

\textbf{Autonomous Lur'e system} is composed by a \textbf{linear system} ($G(s)$) \textbf{controllable} and \textbf{observable} fed by a \textbf{nonlinear function} ($\varphi(e)$) on the feedback error.

\[
    \begin{cases}
        \dx = A x + B \varphi(-Cx) \\
        y = C x
    \end{cases}
\]

The non linear function is bounded into a sector ($\varphi(e) \in \Phi_{[k_1,k_2]}$) defined as

\[
    \Phi_{[k_1,k_2]} = \{\phi(\cdot) : (k_2 e -u)(u - k_1 e) \geq 0, u=\phi(e), \forall e \in \Real \}
\]

For this kind of system $\bar{x} = 0$ is an equilibrium.

\subsection{Meaning}

\begin{figure}[htb]
    \centering
    \resizebox{.9\textwidth}{!}{
    \begin{tikzpicture}
        \node [input] (iy_d) {};

        \node [sum, right=1cm of iy_d] (sum_e) {};
        \node [block, right=0.6cm of sum_e] (controller) {$R(s)$};
        \node [sum, right=0.6cm of controller] (sum_u) {};
        \node [block, right=0.6cm of sum_u] (nonlinearity) {$\Psi(\cdot)$};
        \node [block, right=1cm of nonlinearity] (system) {$P(s)$};
        \node [sum, right=0.6cm of system] (sum_d) {};

        \node [output, right=1cm of sum_d] (oy) {};

        \node [input, above=1cm of sum_u] (iu_n) {};
        \node [input, above=1cm of sum_d] (id_n) {};

        \node [block, below=1cm of nonlinearity] (feedback) {$T(s)$};

        \draw [->] (iy_d) -- node{$y^0$} (sum_e);
        \draw [->] (sum_e) -- node {$e$} (controller);
        \draw [->] (controller) -- node {$w$} (sum_u);
        \draw [->] (sum_u) -- node {$u$} (nonlinearity);
        \draw [->] (nonlinearity) -- node {$m$} (system);
        \draw [->] (system) -- (sum_d);

        \draw [->] (sum_d) -- node [pos=0.8] {$\tilde y$} node [node, name=y, pos=0.5] {} (oy);

        \draw [->] (y) |- (feedback);
        \draw [->] (feedback) -| node[pos=0.07] {$y$} node [pos=0.95] {$-$} (sum_e);

        \draw [->] (iu_n) -- node[pos=0.2] {$u^0$} (sum_u);
        \draw [->] (id_n) -- node[pos=0.2] {$d^0$} (sum_d);

    \end{tikzpicture}
    }
    \caption{A generic feedback system with a nonlinearity}
    \label{fig:generic-feedback-system}
\end{figure}

A general feedback system (e.g \cref{fig:generic-feedback-system}) containing a nonlinear function where input are constant can be traced to an \textbf{autonomous Lur'e system}.

Chosen an equilibrium (associated to a constant inputs) and the whole system can be studied in the neighbourhood of it applying a coordinate shifting (i.e. $\Delta x(t) = x(t) - \bar x$)

From the definition of constant inputs we get $\Delta y^0 = \Delta u^0 = \Delta d^0 = 0$, then we get a system like \cref{fig:generic-feedback-system-equilibrium}, easily to retract back to an \textbf{Autonomous Lur'e system}

\begin{figure}[htb]
    \centering
    \resizebox{.8\textwidth}{!}{
        \begin{tikzpicture}
            \node [sum] (sum_e) {};
            \node [block, right=0.6cm of sum_e] (controller) {$R(s)$};
            \node [block, right=2cm of controller] (nonlinearity) {$\varphi(\cdot)$};
            \node [block, right=1cm of nonlinearity] (system) {$P(s)$};

            \node [output, right=1cm of system] (oy) {};

            \node [block, below=1cm of nonlinearity] (feedback) {$T(s)$};

            \draw [->] (sum_e) -- node {$\Delta e$} (controller);
            \draw [->] (controller) -- node {$\Delta w = \Delta u$} (nonlinearity);
            \draw [->] (nonlinearity) -- node {$\Delta m$} (system);
            \draw [->] (system) -- node [pos=0.8] {$\Delta \tilde y$} node [node, name=y, pos=0.5] {} (oy);

            \draw [->] (y) |- (feedback);
            \draw [->] (feedback) -| node[pos=0.07] {$\Delta y$} node [pos=0.95] {$-$} (sum_e);


        \end{tikzpicture}
    }
    \caption{A generic feedback system with a nonlinearity considered near the equilibrium}
    \label{fig:generic-feedback-system-equilibrium}
\end{figure}

\section{Properties}

\begin{theorem}[Nyquist criterion]\label{thm:nyquist}
A degenerated \textbf{autonomous Lur'e system} with $\varphi(\cdot) = k$ (linearly in the error) is asymptotically stable if and only if the Nyquist plot of the linear part encircles (anti-clockwise) the point of the complex plane $-\frac{1}{k}$ as many times as the amount the unstable poles of it
\end{theorem}

\begin{theorem}\label{thm:lure-nyquist}
    \[
        I(k_1,k_2) = \{\alpha \in \Real : \alpha = -\frac{1}{k}, k \in [k_1,k_2]\}
    \]
    If \textbf{autonomous Lur'e system} is absolutely stable in the sector $[k_1,k_2]$, then the \textbf{Nyquist plot} of the linear part encircles (anti-clockwise) $I(k_1,k_2)$ as many times as the amount the unstable poles of it
\end{theorem}

\begin{nb}\cref{thm:lure-nyquist} defined a necessary condition for stability but not sufficient\end{nb}

\begin{conjecture}[Aizerman conjecture, 1949]\label{cjt:aizerman}
    The \cref{thm:lure-nyquist} is also sufficient for the stability of the \textbf{autonomous Lur'e system}
\end{conjecture}

\begin{nb}
    it is possible find a counterexample for the \cref{cjt:aizerman}
\end{nb}

\subsection{Popov criterion}

\begin{theorem}[Popov criterion, 1962]\label{thm:popov-criterion}
    An \textbf{autonomous Lur'e system} is absolutely stable in sector $[0,k]$ if the linear part is asymptotically stable and exists a real number $q$ such that
    \[
        \Re [(1+\j \omega q) G(\j \omega)] > - \frac{1}{k}, \forall \omega \geq 0
    \]
    is satisfied
\end{theorem}

\begin{nb}\cref{thm:popov-criterion} defined a sufficient condition for stability\end{nb}

\begin{theorem}[Popov criterion (alternative), 1962]
    An \textbf{autonomous Lur'e system} is absolutely stable in sector $[0,k]$ if the linear part is asymptotically stable with transfer function $G(s)$ and exists a positive real number $q$ such that $-\frac{1}{q}$ is not an eigenvalue of the matrix A of the linear part and defined
    \[
        H(s) = 1+k(1+qs)G(s)
    \]
    $H(s)$ is strictly positive real
\end{theorem}

\subsubsection{Special case $q=0$}

With $q=0$ the condition is reduced to

\[
    \Re [G(\j \omega)] > - \frac{1}{k}, \forall \omega \geq 0
\]

so the \cref{thm:popov-criterion} requires that the $G(\j\omega)$ polar plot is contained in the right-half-plane defined by the vertical line passing to the real point $-\frac{1}{k}$

\subsubsection{Case $q \neq 0$}

\begin{corollary}
    Defining
    \[
        G^*(\j\omega) = \Re[G(\j\omega)] + \j\omega \Im[G(\j\omega)]
    \]
    the \cref{thm:popov-criterion} requires that exists a straight line passing through the real point $-\frac{1}{k}$ such that the $G^*(\j\omega)$ polar plot is contained in the right-half-plane defined by it
\end{corollary}

\subsubsection{Popov criterion and Aizenman conjecture}

Let us denote with $K_P$ the largest $K$ value such that the absolute stability of an \textbf{autonomous Lur'e system} in the sector $[0,K]$ is guaranteed via \textbf{Popov criterion};
and with $K_N$ the largest $K$ such that the degenerated \textbf{autonomous Lur'e system} ($\varphi(\cdot)$ linearly in error) is asymptotically stable via \cref{thm:nyquist}(\textbf{Nyquist criterion})

\begin{nb} for an \textbf{autonomous Lur'e system} $K_P \leq K_N$\end{nb}

\begin{nb}if $K_P = K_N$ then, the system satisfies the \textbf{Aizenman Conjecture} \end{nb}

\begin{nb}$K_P < K_N$ does not imply that the system does not satisfy the \textbf{Aizenman Conjecture}\end{nb}

\subsection{Generalized Popov criterion for sector $[k_1,k_2]$}

Consider an \textbf{autonomous Lur'e system} with $\varphi(\cdot) \in \Phi_{[k_1,k_2]}$, then we add and subtract $k_1 e$ from $u$, like in \cref{fig:lure-system-reworked}, notice that the new system is completely equivalent to the original one

\begin{figure}[htb]
    \centering
    \resizebox{0.7\textwidth}{!}{
        \begin{tikzpicture}[node distance = 1cm, auto]
            \tikzset{node distance = 1cm and 2cm}
            \node [sum] (sum_e) {};
            \node [node, right=.6cm of sum_e] (e_2) {};
            \node [block, right=.6cm of e_2] (controller) {$\varphi(\cdot)$};
            \node [block, below=.6cm of controller] (k1_1) {$k_1$};
            \node [sum, right=1cm of controller] (sum_v) {};
            \node [sum, right=1cm of sum_v] (sum_u) {};
            \node [block, right=1cm of sum_u] (system) {$G(s)$};
            \node [block, below=.6cm of system] (k1_2) {$k_1$};
            \node [node, right=.6cm of system] (y_1) {};
            \node [node, right=.6cm of y_1] (y) {};

            \node [output, right=.6cm of y] (oy) {};

            \draw [->] (sum_e) -- node {$e$} (controller);
            \draw [->] (e_2) |- (k1_1);

            \draw [->] (controller) -- (sum_v);
            \draw [->] (k1_1) -| node [pos=0.9] {$-$} (sum_v);
            \draw [->] (sum_v) -- node {$v$} (sum_u);
            \draw [->] (sum_u) -- node {$u$} (system);
            \draw [->] (system) -- node [pos=0.8] {$y$} (oy);
            \draw [->] (y_1) |- (k1_2);
            \draw [->] (k1_2) -| node [pos=0.9] {$-$} (sum_u);

            \draw [->] (y) -- ++(0,-3) -| node [pos=0.95] {$-$} (sum_e);
        \end{tikzpicture}
    }
    \caption{Autonomous Lur'e system reworked}
    \label{fig:lure-system-reworked}
\end{figure}

Now, we rewrite the block diagram in the standard shape of a \textbf{autonomous Lur'e system} like in the \cref{fig:lure-system-reqworked-compact}, where

\begin{gather*}
    F(s) = \frac{G(s)}{1+k_1 G(s)} \\
    \eta(e) = \varphi(e) - k_1 e, \qquad \eta(e) \in \Phi_{[0,k_2-k_1]}
\end{gather*}

\begin{figure}[htb]
    \centering
    \resizebox{0.7\textwidth}{!}{
        \begin{tikzpicture}
            \node [sum] (sum_e) {};

            \node [block, right=1cm of sum_e] (controller) {$\eta(\cdot)$};
            \node [block, right=1cm of controller] (system) {$F(s)$};

            \node [output, right=1cm of system] (oy) {};

            \draw [->] (sum_e) -- node {$e$} (controller);
            \draw [->] (controller) -- node {$u$} (system);
            \draw [->] (system) -- node [pos=0.8] {$y$} node [node, name=y, pos=0.5] {} (oy);

            \draw [->] (y) -- ++(0,-1.5) -| node [pos=0.95] {$-$} (sum_e);
        \end{tikzpicture}
    }
    \caption{Compact autonomous Lur'e system reworked}
    \label{fig:lure-system-reqworked-compact}
\end{figure}

So, due to the equivalence of the original system and the reworked one we can formulate a generalized version of the \textbf{Popov criterion}

\begin{theorem}\label{thm:popov-criterion-generalized}
    An \textbf{autonomous Lur'e system} is absolutely stable in sector $[k_1,k_2]$ if $F(s)$ is asymptotically stable and exists a real number $q$ such that
\[
    \Re \left[(1+\j \omega q) F(s)\right] > - \frac{1}{k_2-k_1}, \forall \omega \geq 0
\]
is satisfied when
\[
    F(s) = \frac{G(s)}{1+k_1 G(s)}
\]
\end{theorem}

\subsubsection{The special case $q=0$}

The condition required by the \cref{thm:popov-criterion-generalized} with an imposed $q=0$ became $F(s)$ asymptotically stable and
\[
    \Re \left[F(s)\right] > - \frac{1}{k_2-k_1}, \; \forall \omega \geq 0
\]

The first condition can be checked remembering that $F(s)$ is a feedback system composed by the loop function $k_1 G(s)$, so exploiting the \textbf{Nyquist criterion} we can assert that $F(s)$ is an asymptotically stable system if the \textbf{Nyquist plot} of $G(s)$ encircles (anti-clockwise) the point $-\frac{1}{k_1}$ as many times the number of unstable poles of $G(s)$.

The second one can be rewrite like a condition on the polar plot of $G(s)$ instead of the one of $F(s)$. Considering that
\[
    G(s) = \frac{F(s)}{1 - k_1 F(s)}
\]

we can map the right-half-plane ($>-\frac{1}{k_2 - k_1}$) in which the polar plot of $F(s)$ must be contained, to a circle where the polar plot of $G(s)$ must not go in. This circle is defined as

\[
    O(k_1,k_2) = \{\tilde{s} \in \Complex : \tilde{s} = \frac{s}{1-k_1 s}, s = \alpha + \j \beta, \alpha \leq - \frac{1}{k_2 - k_1}\}
\]

Merging the two conditions we can formulate the \textbf{Circle criterion}

\begin{theorem}[Circle criterion]
    An \textbf{autonomous Lur'e system} is absolutely stable in sector $[k_1,k_2]$ if the \textbf{Nyquist plot} of $G(s)$ encircles (anti-clockwise) the circle $O(k_1, k_2)$ as many time the number of unstable poles of $G(s)$, with
    \[
        O(k_1,k_2) = \{\tilde{s} \in \Complex : \tilde{s} = \frac{s}{1-k_1 s}, s = \alpha + \j \beta, \alpha \leq - \frac{1}{k_2 - k_1}\}
    \]
\end{theorem}

\printindex

\end{document}