\documentclass[12pt]{report}

\usepackage[utf8]{inputenc}
\usepackage{amssymb}
\usepackage{amsmath}
\usepackage{amsthm}
\usepackage[makeroom]{cancel}
\usepackage{svg}
\usepackage{tabularx}
\usepackage{soul}
\usepackage[pdfusetitle]{hyperref}
\usepackage{enumitem}
\usepackage{nicefrac}
\usepackage[capitalize,nameinlink,noabbrev]{cleveref}
\usepackage{caption}
\usepackage{subcaption}

\usepackage{geometry}
\geometry{
a4paper,
left=30mm,
right=30mm,
top=40mm,
bottom=30mm
}

\setcounter{tocdepth}{3}
\setcounter{secnumdepth}{3}

\usepackage{tikz}
\usetikzlibrary{circuits}
\usetikzlibrary{shapes,arrows}
\usetikzlibrary{positioning}

\tikzset{auto,>=latex', minimum size=0pt, node distance=1, font=\fontsize{11}{11}\selectfont}

\tikzstyle{block} = [draw, fill=white, rectangle, minimum height=3em, minimum width=4em, text centered, align=center]
\tikzstyle{integrator} = [draw, fill=white, rectangle, minimum height=3em, minimum width=2em, font={$\frac{1}{s}$}]
\tikzstyle{sum} = [draw, fill=white, circle, node distance=1cm]
\tikzstyle{input} = [coordinate]
\tikzstyle{output} = [coordinate]
\tikzstyle{pinstyle} = [pin edge={to-,thin,black}]
\tikzstyle{gain} = [draw, isosceles triangle, minimum height = 3em, isosceles triangle apex angle=60]
\tikzstyle{spy} = [coordinate, inner sep=0, outer sep=0, minimum size=0]
\tikzstyle{node} = [draw, circle, fill, minimum size=2pt, inner sep=0pt, outer sep=0pt, anchor=center]

%\usepackage{bodegraph}

%\usetikzlibrary{intersections}
%\usetikzlibrary{calc}

\setlength{\parindent}{0em}
\setlength{\parskip}{0.5em}

\newenvironment{conditions}
{\par\vspace{\abovedisplayskip}\noindent\centering\begin{tabular}{>{$}r<{$} @{${}:{}$} l}}
{\end{tabular}\par\vspace{\belowdisplayskip}}

\newenvironment{nb}{\par\vspace{\abovedisplayskip}\noindent\begin{em}n.b.\ \ignorespaces}{\end{em}}

\newcommand{\pd}[2]{\frac{\partial #1}{\partial #2}}
\newcommand{\diff}[2]{\frac{d #1}{d #2}}


%\DeclareMathOperator{\Re}{Re}
%\DeclareMathOperator{\Im}{Im}
\DeclareMathOperator{\rad}{rad}
\DeclareMathOperator{\sgn}{sgn}
\DeclareMathOperator{\diag}{diag}

\newcommand{\matr}[1]{\boldsymbol{#1}}
\newcommand{\vect}[1]{\boldsymbol{#1}}

\newcommand{\trans}{\mathsf{T}}
\newcommand{\inv}{{-1}}
\newcommand{\psinv}{\dag}
\newcommand\norm[1]{\left\lVert#1\right\rVert}

\newcommand{\Real}{\mathbb{R}}
\newcommand{\Complex}{\mathbb{C}}

\newcommand{\I}{\matr{I}}
\newcommand{\0}{\vect{0}}

\newcommand{\J}{\matr{J}}
\newcommand{\dJ}{\dot{\J}}

\newcommand{\estimate}[1]{\hat{#1}}
\newcommand{\error}[1]{\tilde{#1}}
\newcommand{\reference}[1]{\bar{#1}}

\newcommand{\B}{\matr{B}}
\newcommand{\dB}{\dot\B}
\newcommand{\Bs}{\hat\B}
\newcommand{\Be}{\error\B}

\newcommand{\C}{\matr{C}}
\newcommand{\Cs}{\estimate\C}
\newcommand{\Ce}{\error\C}

\newcommand{\g}{\vect{g}}
\newcommand{\gs}{\estimate\g}

\newcommand{\n}{\vect{n}}
\newcommand{\ns}{\estimate\n}
\newcommand{\ner}{\error\n} % #r becuase ne already exists

\newcommand{\K}{\matr{K}}

\newcommand{\y}{\vect{y}}
\newcommand{\dy}{\dot{\y}}
\newcommand{\ddy}{\ddot{\y}}

\newcommand{\V}{V}
\newcommand{\dV}{\dot{\V}}

\newcommand{\q}{\vect{q}}
\newcommand{\dq}{\dot{\q}}
\newcommand{\ddq}{\ddot{\q}}
\newcommand{\qd}{\bar{\q}}
\newcommand{\dqd}{\dot{\reference{\q}}}
\newcommand{\ddqd}{\ddot{\reference{\q}}}
\newcommand{\qe}{\error{\q}}
\newcommand{\dqe}{\dot{\error{\q}}}
\newcommand{\ddqe}{\ddot{\error{\q}}}
\newcommand{\qs}{\estimate{\q}}
\newcommand{\dqs}{\dot{\estimate{\q}}}
\newcommand{\ddqs}{\ddot{\estimate{\q}}}

\newcommand{\x}{\vect x}
\newcommand{\dx}{\dot{\x}}
\newcommand{\ddx}{\ddot{\x}}
\newcommand{\xd}{\reference{\x}}
\newcommand{\dxd}{\dot{\reference{\x}}}
\newcommand{\ddxd}{\ddot{\reference{\x}}}
\newcommand{\xe}{\error{\x}}
\newcommand{\dxe}{\dot{\error{\x}}}
\newcommand{\ddxe}{\ddot{\error{\x}}}
\newcommand{\xs}{\estimate{\x}}
\newcommand{\dxs}{\dot{\estimate{\x}}}
\newcommand{\ddxs}{\ddot{\estimate{\x}}}

\newcommand{\e}{\vect e}
\newcommand{\de}{\dot{\e}}

\newcommand{\Lie}[1]{\mathcal{L}_{#1}}

\newtheorem{theorem}{Theorem}[chapter]
\newtheorem{corollary}{Corollary}[theorem]
\newtheorem{lemma}[theorem]{Lemma}
\newtheorem{conjecture}{Conjecture}[chapter]
\newtheorem{example}{Example}[chapter]
\newtheorem{definition}{Definition}[chapter]

\newcommand{\warning}[1]{\textcolor{red}{WARNING #1}}

\usepackage{imakeidx}
\makeindex[intoc]

\usepackage[backend=biber]{biblatex}
\addbibresource{references.bib}

\title{Nonlinear control\\Notes\thanks{the whole document is based on \cite{nc-slides}}}
\author{Roberto Bochet}

\begin{document}

\maketitle

\printbibliography

\tableofcontents

\chapter{Passive systems}

\section{Definition}

\begin{gather*}
    \begin{cases}
        \dx (t) = f(\x,u) \\
        y(t) = g(\x,u)
    \end{cases} \\
    f(\vect 0,0) = \vect 0, \quad g(\vect 0,0)=0
\end{gather*}

A dynamical system is called \textbf{passive} if there exists a \textbf{Lyapunov function} $\V(\x)$ called \textbf{storage function}, such that

\[
    uy \geq \pd{V}{\x}(\x) f(\x,u) + \epsilon u^2 + \delta y^2 + \rho \psi(\x)
\]

with

\[
    \epsilon \geq 0, \delta \geq 0, \rho \geq 0, \psi(\x) > 0
\]

Notice that the Lyapunov function can be written as

\[
     \pd{\V}{\x} f(\x,0) = \dV(\x)
\]

\subsection{Conservative system}

A passive system is called \textbf{conservative} if $\epsilon = \delta = \rho = 0$, such that

\[
    uy = \pd{V}{\x}(\x) f(\x,u)
\]

\subsection{Strictly passive}

A passive system is called strictly passive if any of $\epsilon, \delta, \rho$ is greater that $0$.
In particular

Strictly passive w.r.t the input if $\epsilon > 0$ \\
Strictly passive w.r.t the output if $\delta > 0$ \\
Strictly passive w.r.t the state if $\rho > 0$

\subsection{Zero-state observable}

A system is \textbf{zero-state observable} if $\x(t) = 0$ is the only \textbf{free evolution} (u=0) of the state compatible with identically zero output (y=0)

\section{Properties}

\begin{theorem}
    If a system is \textbf{passive} with positive definite storage function, then $\x=0$ is a \textbf{stable} equilibrium for the system with zero input
\end{theorem}

\begin{theorem}
    If a system is \textbf{strictly passive w.r.t the state} with positive definite storage function, then $\x=0$ is an \textbf{asymptotically stable} equilibrium for the system with zero input
\end{theorem}

\begin{theorem}
    If a system is \textbf{strictly passive w.r.t the output} and \textbf{zero-stable observable} with positive definite storage function, then $\x=0$ is an \textbf{asymptotically stable} equilibrium for the system with zero input
\end{theorem}

\begin{theorem}
    If the storage function associated to \textbf{asymptotically stable} equilibrium is radially unbounded then the equilibrium is \textbf{GAS}
\end{theorem}

\begin{theorem}
    If a system is \textbf{zero state observable} and \textbf{passive} with positive definite storage function and \textbf{radially unbounded}, then $\x=0$ is a \textbf{GAS equilibrium} of the output feedback system with control law
    \[
        u=-\phi(y),\qquad \phi(0)=0, \quad y\phi(y)>0, \forall y \neq 0
    \]
\end{theorem}
\chapter{Lur'e systems}

\section{Definition}

\begin{figure}[htb]
    \centering
    \resizebox{0.7\textwidth}{!}{
        \begin{tikzpicture}
            \node [sum] (sum_e) {};

            \node [block, right=1cm of sum_e] (controller) {$\varphi(\cdot)$};
            \node [block, right=1cm of controller] (system) {$G(s)$};

            \node [output, right=1cm of system] (oy) {};

            \draw [->] (sum_e) -- node {$e$} (controller);
            \draw [->] (controller) -- node {$u$} (system);
            \draw [->] (system) -- node [pos=0.8] {$y$} node [node, name=y, pos=0.5] {} (oy);

            \draw [->] (y) -- ++(0,-1.5) -| node [pos=0.95] {$-$} (sum_e);
        \end{tikzpicture}
    }
    \caption{Autonomous Lur'e system}
    \label{fig:lure-system}
\end{figure}

\textbf{Autonomous Lur'e system} is composed by a \textbf{linear system} ($G(s)$) \textbf{controllable} and \textbf{observable} fed by a \textbf{nonlinear function} ($\varphi(e)$) on the feedback error.

\[
    \begin{cases}
        \dx = A x + B \varphi(-Cx) \\
        y = C x
    \end{cases}
\]

The non linear function is bounded into a sector ($\varphi(e) \in \Phi_{[k_1,k_2]}$) defined as

\[
    \Phi_{[k_1,k_2]} = \{\phi(\cdot) : (k_2 e -u)(u - k_1 e) \geq 0, u=\phi(e), \forall e \in \Real \}
\]

For this kind of system $\bar{x} = 0$ is an equilibrium.

\subsection{Meaning}

\begin{figure}[htb]
    \centering
    \resizebox{.9\textwidth}{!}{
    \begin{tikzpicture}
        \node [input] (iy_d) {};

        \node [sum, right=1cm of iy_d] (sum_e) {};
        \node [block, right=0.6cm of sum_e] (controller) {$R(s)$};
        \node [sum, right=0.6cm of controller] (sum_u) {};
        \node [block, right=0.6cm of sum_u] (nonlinearity) {$\Psi(\cdot)$};
        \node [block, right=1cm of nonlinearity] (system) {$P(s)$};
        \node [sum, right=0.6cm of system] (sum_d) {};

        \node [output, right=1cm of sum_d] (oy) {};

        \node [input, above=1cm of sum_u] (iu_n) {};
        \node [input, above=1cm of sum_d] (id_n) {};

        \node [block, below=1cm of nonlinearity] (feedback) {$T(s)$};

        \draw [->] (iy_d) -- node{$y^0$} (sum_e);
        \draw [->] (sum_e) -- node {$e$} (controller);
        \draw [->] (controller) -- node {$w$} (sum_u);
        \draw [->] (sum_u) -- node {$u$} (nonlinearity);
        \draw [->] (nonlinearity) -- node {$m$} (system);
        \draw [->] (system) -- (sum_d);

        \draw [->] (sum_d) -- node [pos=0.8] {$\tilde y$} node [node, name=y, pos=0.5] {} (oy);

        \draw [->] (y) |- (feedback);
        \draw [->] (feedback) -| node[pos=0.07] {$y$} node [pos=0.95] {$-$} (sum_e);

        \draw [->] (iu_n) -- node[pos=0.2] {$u^0$} (sum_u);
        \draw [->] (id_n) -- node[pos=0.2] {$d^0$} (sum_d);

    \end{tikzpicture}
    }
    \caption{A generic feedback system with a nonlinearity}
    \label{fig:generic-feedback-system}
\end{figure}

A general feedback system (e.g \cref{fig:generic-feedback-system}) containing a nonlinear function where input are constant can be traced to an \textbf{autonomous Lur'e system}.

Chosen an equilibrium (associated to a constant inputs) and the whole system can be studied in the neighbourhood of it applying a coordinate shifting (i.e. $\Delta x(t) = x(t) - \bar x$)

From the definition of constant inputs we get $\Delta y^0 = \Delta u^0 = \Delta d^0 = 0$, then we get a system like \cref{fig:generic-feedback-system-equilibrium}, easily to retract back to an \textbf{Autonomous Lur'e system}

\begin{figure}[htb]
    \centering
    \resizebox{.8\textwidth}{!}{
        \begin{tikzpicture}
            \node [sum] (sum_e) {};
            \node [block, right=0.6cm of sum_e] (controller) {$R(s)$};
            \node [block, right=2cm of controller] (nonlinearity) {$\varphi(\cdot)$};
            \node [block, right=1cm of nonlinearity] (system) {$P(s)$};

            \node [output, right=1cm of system] (oy) {};

            \node [block, below=1cm of nonlinearity] (feedback) {$T(s)$};

            \draw [->] (sum_e) -- node {$\Delta e$} (controller);
            \draw [->] (controller) -- node {$\Delta w = \Delta u$} (nonlinearity);
            \draw [->] (nonlinearity) -- node {$\Delta m$} (system);
            \draw [->] (system) -- node [pos=0.8] {$\Delta \tilde y$} node [node, name=y, pos=0.5] {} (oy);

            \draw [->] (y) |- (feedback);
            \draw [->] (feedback) -| node[pos=0.07] {$\Delta y$} node [pos=0.95] {$-$} (sum_e);


        \end{tikzpicture}
    }
    \caption{A generic feedback system with a nonlinearity considered near the equilibrium}
    \label{fig:generic-feedback-system-equilibrium}
\end{figure}

\section{Properties}

\begin{theorem}[Nyquist criterion]\label{thm:nyquist}
A degenerated \textbf{autonomous Lur'e system} with $\varphi(\cdot) = k$ (linearly in the error) is asymptotically stable if and only if the Nyquist plot of the linear part encircles (anti-clockwise) the point of the complex plane $-\frac{1}{k}$ as many times as the amount the unstable poles of it
\end{theorem}

\begin{theorem}\label{thm:lure-nyquist}
    \[
        I(k_1,k_2) = \{\alpha \in \Real : \alpha = -\frac{1}{k}, k \in [k_1,k_2]\}
    \]
    If \textbf{autonomous Lur'e system} is absolutely stable in the sector $[k_1,k_2]$, then the \textbf{Nyquist plot} of the linear part encircles (anti-clockwise) $I(k_1,k_2)$ as many times as the amount the unstable poles of it
\end{theorem}

\begin{nb}\cref{thm:lure-nyquist} defined a necessary condition for stability but not sufficient\end{nb}

\begin{conjecture}[Aizerman conjecture, 1949]\label{cjt:aizerman}
    The \cref{thm:lure-nyquist} is also sufficient for the stability of the \textbf{autonomous Lur'e system}
\end{conjecture}

\begin{nb}
    it is possible find a counterexample for the \cref{cjt:aizerman}
\end{nb}

\subsection{Popov criterion}

\begin{theorem}[Popov criterion, 1962]\label{thm:popov-criterion}
    An \textbf{autonomous Lur'e system} is absolutely stable in sector $[0,k]$ if the linear part is asymptotically stable and exists a real number $q$ such that
    \[
        \Re [(1+\j \omega q) G(\j \omega)] > - \frac{1}{k}, \forall \omega \geq 0
    \]
    is satisfied
\end{theorem}

\begin{nb}\cref{thm:popov-criterion} defined a sufficient condition for stability\end{nb}

\begin{theorem}[Popov criterion (alternative), 1962]
    An \textbf{autonomous Lur'e system} is absolutely stable in sector $[0,k]$ if the linear part is asymptotically stable with transfer function $G(s)$ and exists a positive real number $q$ such that $-\frac{1}{q}$ is not an eigenvalue of the matrix A of the linear part and defined
    \[
        H(s) = 1+k(1+qs)G(s)
    \]
    $H(s)$ is strictly positive real
\end{theorem}

\subsubsection{Special case $q=0$}

With $q=0$ the condition is reduced to

\[
    \Re [G(\j \omega)] > - \frac{1}{k}, \forall \omega \geq 0
\]

so the \cref{thm:popov-criterion} requires that the $G(\j\omega)$ polar plot is contained in the right-half-plane defined by the vertical line passing to the real point $-\frac{1}{k}$

\subsubsection{Case $q \neq 0$}

\begin{corollary}
    Defining
    \[
        G^*(\j\omega) = \Re[G(\j\omega)] + \j\omega \Im[G(\j\omega)]
    \]
    the \cref{thm:popov-criterion} requires that exists a straight line passing through the real point $-\frac{1}{k}$ such that the $G^*(\j\omega)$ polar plot is contained in the right-half-plane defined by it
\end{corollary}

\subsubsection{Popov criterion and Aizenman conjecture}

Let us denote with $K_P$ the largest $K$ value such that the absolute stability of an \textbf{autonomous Lur'e system} in the sector $[0,K]$ is guaranteed via \textbf{Popov criterion};
and with $K_N$ the largest $K$ such that the degenerated \textbf{autonomous Lur'e system} ($\varphi(\cdot)$ linearly in error) is asymptotically stable via \cref{thm:nyquist}(\textbf{Nyquist criterion})

\begin{nb} for an \textbf{autonomous Lur'e system} $K_P \leq K_N$\end{nb}

\begin{nb}if $K_P = K_N$ then, the system satisfies the \textbf{Aizenman Conjecture} \end{nb}

\begin{nb}$K_P < K_N$ does not imply that the system does not satisfy the \textbf{Aizenman Conjecture}\end{nb}

\subsection{Generalized Popov criterion for sector $[k_1,k_2]$}

Consider an \textbf{autonomous Lur'e system} with $\varphi(\cdot) \in \Phi_{[k_1,k_2]}$, then we add and subtract $k_1 e$ from $u$, like in \cref{fig:lure-system-reworked}, notice that the new system is completely equivalent to the original one

\begin{figure}[htb]
    \centering
    \resizebox{0.7\textwidth}{!}{
        \begin{tikzpicture}[node distance = 1cm, auto]
            \tikzset{node distance = 1cm and 2cm}
            \node [sum] (sum_e) {};
            \node [node, right=.6cm of sum_e] (e_2) {};
            \node [block, right=.6cm of e_2] (controller) {$\varphi(\cdot)$};
            \node [block, below=.6cm of controller] (k1_1) {$k_1$};
            \node [sum, right=1cm of controller] (sum_v) {};
            \node [sum, right=1cm of sum_v] (sum_u) {};
            \node [block, right=1cm of sum_u] (system) {$G(s)$};
            \node [block, below=.6cm of system] (k1_2) {$k_1$};
            \node [node, right=.6cm of system] (y_1) {};
            \node [node, right=.6cm of y_1] (y) {};

            \node [output, right=.6cm of y] (oy) {};

            \draw [->] (sum_e) -- node {$e$} (controller);
            \draw [->] (e_2) |- (k1_1);

            \draw [->] (controller) -- (sum_v);
            \draw [->] (k1_1) -| node [pos=0.9] {$-$} (sum_v);
            \draw [->] (sum_v) -- node {$v$} (sum_u);
            \draw [->] (sum_u) -- node {$u$} (system);
            \draw [->] (system) -- node [pos=0.8] {$y$} (oy);
            \draw [->] (y_1) |- (k1_2);
            \draw [->] (k1_2) -| node [pos=0.9] {$-$} (sum_u);

            \draw [->] (y) -- ++(0,-3) -| node [pos=0.95] {$-$} (sum_e);
        \end{tikzpicture}
    }
    \caption{Autonomous Lur'e system reworked}
    \label{fig:lure-system-reworked}
\end{figure}

Now, we rewrite the block diagram in the standard shape of a \textbf{autonomous Lur'e system} like in the \cref{fig:lure-system-reqworked-compact}, where

\begin{gather*}
    F(s) = \frac{G(s)}{1+k_1 G(s)} \\
    \eta(e) = \varphi(e) - k_1 e, \qquad \eta(e) \in \Phi_{[0,k_2-k_1]}
\end{gather*}

\begin{figure}[htb]
    \centering
    \resizebox{0.7\textwidth}{!}{
        \begin{tikzpicture}
            \node [sum] (sum_e) {};

            \node [block, right=1cm of sum_e] (controller) {$\eta(\cdot)$};
            \node [block, right=1cm of controller] (system) {$F(s)$};

            \node [output, right=1cm of system] (oy) {};

            \draw [->] (sum_e) -- node {$e$} (controller);
            \draw [->] (controller) -- node {$u$} (system);
            \draw [->] (system) -- node [pos=0.8] {$y$} node [node, name=y, pos=0.5] {} (oy);

            \draw [->] (y) -- ++(0,-1.5) -| node [pos=0.95] {$-$} (sum_e);
        \end{tikzpicture}
    }
    \caption{Compact autonomous Lur'e system reworked}
    \label{fig:lure-system-reqworked-compact}
\end{figure}

So, due to the equivalence of the original system and the reworked one we can formulate a generalized version of the \textbf{Popov criterion}

\begin{theorem}\label{thm:popov-criterion-generalized}
    An \textbf{autonomous Lur'e system} is absolutely stable in sector $[k_1,k_2]$ if $F(s)$ is asymptotically stable and exists a real number $q$ such that
\[
    \Re \left[(1+\j \omega q) F(s)\right] > - \frac{1}{k_2-k_1}, \forall \omega \geq 0
\]
is satisfied when
\[
    F(s) = \frac{G(s)}{1+k_1 G(s)}
\]
\end{theorem}

\subsubsection{The special case $q=0$}

The condition required by the \cref{thm:popov-criterion-generalized} with an imposed $q=0$ became $F(s)$ asymptotically stable and
\[
    \Re \left[F(s)\right] > - \frac{1}{k_2-k_1}, \; \forall \omega \geq 0
\]

The first condition can be checked remembering that $F(s)$ is a feedback system composed by the loop function $k_1 G(s)$, so exploiting the \textbf{Nyquist criterion} we can assert that $F(s)$ is an asymptotically stable system if the \textbf{Nyquist plot} of $G(s)$ encircles (anti-clockwise) the point $-\frac{1}{k_1}$ as many times the number of unstable poles of $G(s)$.

The second one can be rewrite like a condition on the polar plot of $G(s)$ instead of the one of $F(s)$. Considering that
\[
    G(s) = \frac{F(s)}{1 - k_1 F(s)}
\]

we can map the right-half-plane ($>-\frac{1}{k_2 - k_1}$) in which the polar plot of $F(s)$ must be contained, to a circle where the polar plot of $G(s)$ must not go in. This circle is defined as

\[
    O(k_1,k_2) = \{\tilde{s} \in \Complex : \tilde{s} = \frac{s}{1-k_1 s}, s = \alpha + \j \beta, \alpha \leq - \frac{1}{k_2 - k_1}\}
\]

Merging the two conditions we can formulate the \textbf{Circle criterion}

\begin{theorem}[Circle criterion]
    An \textbf{autonomous Lur'e system} is absolutely stable in sector $[k_1,k_2]$ if the \textbf{Nyquist plot} of $G(s)$ encircles (anti-clockwise) the circle $O(k_1, k_2)$ as many time the number of unstable poles of $G(s)$, with
    \[
        O(k_1,k_2) = \{\tilde{s} \in \Complex : \tilde{s} = \frac{s}{1-k_1 s}, s = \alpha + \j \beta, \alpha \leq - \frac{1}{k_2 - k_1}\}
    \]
\end{theorem}
\chapter{Variable structure control}

\section{Definition}

\begin{figure}[htb]
    \centering
    \resizebox{0.7\textwidth}{!}{
        \begin{tikzpicture}
            \node [sum] (sum_e) {};

            \node [block, right=1cm of sum_e] (controller) {$\psi(\cdot)$};
            \node [block, above=1cm of controller] (switching_law) {$s(\cdot) > 0$};
            \node [block, right=1cm of controller] (system) {$G(s)$};

            \node [output, right=1cm of system] (oy) {};

            \draw [->] (sum_e) -- node {$e$} (controller);
            \draw [->] (switching_law) -- (controller);
            \draw [->] (controller) -- node {$u$} (system);
            \draw [->] (system) |- node [pos=0.8] {$x$} (switching_law);
            \draw [->] (system) -- node [pos=0.8] {$y$} node [node, name=y, pos=0.5] {} (oy);

            \draw [->] (y) -- ++(0,-1.5) -| node [pos=0.95] {$-$} (sum_e);
        \end{tikzpicture}
    }
    \caption{Example of VSC}
    \label{fig:vsc-example}
\end{figure}

In \cref{fig:vsc-example} is shown an example of the structure of a typical \textbf{VSC}, in particular we have a piecewise control law, based on the switching function $s(x) > 0$.

\[
    \psi(e) = \begin{cases}
                  \varphi_1(e), \qquad s(x) > 0 \\
                  \varphi_2(e), \qquad s(x) < 0
              \end{cases}
\]

given this control law we get two different overall systems, one based on the control law for $s(x)>0$ and one for $s(x)<0$ (\cref{fig:vcs-two-systems}).

\begin{figure}[htb]
    \centering
    \begin{subfigure}[b]{0.45\textwidth}
        \resizebox{\textwidth}{!}{
            \begin{tikzpicture}
                \node [sum] (sum_e) {};

                \node [block, right=1cm of sum_e] (controller) {$\varphi_1(\cdot)$};
                \node [block, right=1cm of controller] (system) {$G(s)$};

                \node [output, right=1cm of system] (oy) {};

                \draw [->] (sum_e) -- node {$e$} (controller);
                \draw [->] (controller) -- node {$u$} (system);
                \draw [->] (system) -- node [pos=0.8] {$y$} node [node, name=y, pos=0.5] {} (oy);

                \draw [->] (y) -- ++(0,-1.5) -| node [pos=0.95] {$-$} (sum_e);
            \end{tikzpicture}
        }
        \caption{$s(x)>0$}
        \label{fig:vcs-two-systems-a}
    \end{subfigure}
    \hfill
    \begin{subfigure}[b]{0.45\textwidth}
        \resizebox{\textwidth}{!}{
            \begin{tikzpicture}
                \node [sum] (sum_e) {};

                \node [block, right=1cm of sum_e] (controller) {$\varphi_2(\cdot)$};
                \node [block, right=1cm of controller] (system) {$G(s)$};

                \node [output, right=1cm of system] (oy) {};

                \draw [->] (sum_e) -- node {$e$} (controller);
                \draw [->] (controller) -- node {$u$} (system);
                \draw [->] (system) -- node [pos=0.8] {$y$} node [node, name=y, pos=0.5] {} (oy);

                \draw [->] (y) -- ++(0,-1.5) -| node [pos=0.95] {$-$} (sum_e);
            \end{tikzpicture}
        }
        \caption{$s(x)<0$}
        \label{fig:vcs-two-systems-b}
    \end{subfigure}
    \caption{The two systems under the switching law}
    \label{fig:vcs-two-systems}
\end{figure}

\begin{nb}the stability of the two systems tells us nothing about the stability of the overall one\end{nb}

Considering the vector fields drive the evolution of the states of the two systems (\cref{fig:vcs-two-systems}), the switching law defines a new piecewise vector field merging these two.

The inside boundaries of the new piecewise vector field can be of two kinds:

\begin{itemize}
    \item Across switching surface

    the state reaches the boundary while following some dynamics, crosses it, and continues its evolution according to the other dynamics

    \item Attractive switching surface

    the state reaches the boundary and cannot leave it because the vector fields on both sides are pointing towards the boundary;
    the state can evolve only sliding along the boundary, the system lose a dof (sliding mode)
\end{itemize}

\section{Design}

Given a linear time-invariant SISO system of order $n$ with $(A,B)$ controllable and $(A,C)$ observable, we want design a variable structure controller such that $y(t)$ tends to some (constant) reference signal $\bar y^0$ in some reasonable amount of time, for all $\bar y^0$ and for all possible initial state.

Suppose to put in the controllable canonical form the linear system, so the system can be define as

\[
    \begin{cases}
        \dx_i = x_{i+1}, \qquad i = 1,2,\dots, n-1 \\
        \dx_n = - a_n x_1 - a_{n-1} x_2 - \dots - a_1 x_n + u \\
        y = b_n x_1 + b_{n-1} x_2 + \dots + b_1 x_n
    \end{cases}
\]

\subsection{Design the switching function}

A possible choice for the switching function can be

\[
    s(\vect x) = \beta_{n-1} x_1 + \beta_{n-2} x_2 + \dots + \beta_1 x_{n-1} + x_n - \bar w
\]

on the boundary $s(\vect x)=0$ the system is subjected to the constraint

\[
    x_n = - \beta_{n-1} x_1 - \beta_{n-2} x_2 - \dots - \beta_1 x_{n-1} + \bar w
\]

so, the overall system became

\[
    \begin{cases}
        \dx_i = x_{i+1}, \qquad i = 1,2,\dots, n-2 \\
        \dx_{n-1} = x_n = - \beta_{n-1} x_1 - \beta_{n-2} x_2 - \dots - \beta_1 x_{n-1} + \bar w \\
        y = b_n x_1 + b_{n-1} x_2 + \dots + b_1 x_n
    \end{cases}
\]

this system has the characteristic polynomial defined as

\[
    \chi(s) = s^{n-1} + \beta_1 s^{n-2} + \dots + \beta_{n-1}
\]

which roots can be arbitrarily assigned such that the system became asymptotically stable with a single equilibrium defined as

\[
    \begin{cases}
        \bar x_i = 0, \qquad i = 2,3,\dots, n-1 \\
        \bar x_1 = \frac{\bar w}{\beta_{n-1}} \\
        \bar y = b_n \bar x_1
    \end{cases}
\]

and imposing $\bar w = \frac{\beta_{n-1}}{b_n} \bar y^0$ we get $\bar y = \bar y^0$ and the switching law become

\begin{equation}
    s(\vect x) = \vect \beta^\trans \vect x - \gamma \bar y^0, \qquad
    \vect \beta^\trans = \begin{bmatrix}
                             \beta_{n-1}, \beta_{n-2}, \dots,  \beta_1, 1
    \end{bmatrix},
    \gamma = \frac{\beta_{n-1}}{b_n}
    \label{eqn:switching-law}
\end{equation}

So, the switching law drives the state from any point of $s(\vect x)=0$ to the equilibrium $\bar y = \bar y^0$ following an arbitrary dynamic.

\subsection{Design the control law}

The switching law will drive the state from any point $s(\vect x) = 0$ to the arbitrarily equilibrium, so the control law $u(\vect x)$ will have to drive the state from any point of the vector space to a point of the boundary $s(\vect x) = 0$ in finite time.

The goal: impose the dynamics of $s(\vect x)$ such that from any point of vector field the state reaches $s(\vect x) = 0$.

We shall adopt the so-called "reaching-law approach" to impose the reaching condition.

Defining the dynamics of the switching law such that Lyapunov function

\[
    V(s) = \frac{1}{2} s^2
\]

has negative time derivative satisfying

\[
    \diff{V}{t} = s \dot s \leq - \eta |s|, \quad \eta > 0
\]

Let us propose the dynamics

\[
    \dot s = -q \sgn(s) - r g(s), \qquad q> 0, r \geq 0, g(\cdot) : sg(s) > 0, \forall s \neq 0
\]

for which

\[
    \diff{V}{t} = s \dot s = -q \sgn(s) s - r s g(s) = -q |s| - r s g(s) \leq - q |s|
\]

that satisfy the reaching condition in finite time (with $\eta = q$).

\subsubsection{Impose the switching law dynamics}

Differencing with respect the time the \cref{eqn:switching-law} we get

\[
    \dot s = \vect \beta^\trans \dot{\vect x} = \vect \beta^\trans (A \vect x + B u)
\]

\begin{align*}
    \vect \beta^\trans (A \vect x + B u) &= -q \sgn(s) - r g(s) \\
    u &= - \vect \beta^\trans A \vect x -q \sgn(s) - r g(s) \\
    u &= - \vect \beta^\trans A \vect x -q \sgn(\vect \beta^\trans \vect x - \gamma \bar y^0) - r g(\vect \beta^\trans \vect x - \gamma \bar y^0) \\
    u &= - \vect \beta^\trans A \vect x + q \sgn(\gamma \bar y^0 - \vect \beta^\trans \vect x) - r g(\vect \beta^\trans \vect x - \gamma \bar y^0) \\
    \intertext{$g(s)$ is a odd function from definition, $g(-s)=-g(s)$}
    u &= - \vect \beta^\trans A \vect x + q \sgn(\gamma \bar y^0 - \vect \beta^\trans \vect x) + r g(\gamma \bar y^0 - \vect \beta^\trans \vect x)
\end{align*}

The resulting control system are shown in \cref{fig:vsc-control}.

The control law implements a 2-level hysteresis switching controller.

\begin{nb}hysteresis avoids infinitely fast switching\end{nb}

\begin{figure}[htb]
    \centering
    \resizebox{\textwidth}{!}{
    \begin{tikzpicture}
        \node [input] (iy_d) {};

        \node [block, right=0.6cm of iy_d] (gamma) {$\gamma$};
        \node [sum, right=.6cm of gamma] (sum_e) {};
        \node [block, right=1cm of sum_e] (sgn) {$\sgn(\cdot)$};
        \node [block, above=1cm of sgn] (g) {$g(\cdot)$};
        \node [block, right=0.6cm of sgn] (q) {$q$};
        \node [block, above=1cm of q] (r) {$r$};

        \node [block, below=2cm of sgn] (beta) {$\vect \beta^\trans$};

        \node [sum, right=0.6cm of q] (sum_u) {};
        \node [block, right=1cm of sum_u] (system) {$G(s)$};
        \node [block, below=.8cm of system] (alpha) {$\vect \beta^\trans A$};

        \node [output, right=1cm of system, yshift=0.3cm] (oy) {};

        \node [node, right=.5 of alpha] (xalpha) {};

        \draw [->] (iy_d) -- node[pos=0.2] {$\bar y^0$} (gamma);

        \draw [->] (gamma) -- (sum_e);
        \draw [->] (sum_e) -- node[node, name=s2, pos=.5] {} node[swap]{$-s$} (sgn);
        \draw [->] (s2) |- (g);
        \draw [->] (sgn) -- (q);
        \draw [->] (g) -- (r);
        \draw [->] (q) -- (sum_u);
        \draw [->] (r) -| (sum_u);
        \draw [->] (sum_u) -- node {$u$} (system);

        \draw [->] (alpha) -| node[pos=0.9] {$-$} (sum_u);

        \draw [->] (system)[yshift=-.3cm] -| node[pos=0.6] {$x$} (xalpha) |- (beta);
        \draw [->] (xalpha) -- (alpha);

        \draw [->] (beta) -| node[pos=0.9] {$-$} (sum_e);

        \draw [->] (system.east |- oy) -- node [pos=0.7] {$y$} (oy);
    \end{tikzpicture}
    }
    \caption{Designed control law}
    \label{fig:vsc-control}
\end{figure}
\usepackage{amsmath}\chapter{Feedback linearization}

\section{Definition}

Given a nonlinear system in the form

\[
    \begin{cases}
        \dx =\vect a(\x) + \vect b(\x) u \\
        y = \vect c(\x)
    \end{cases}
\]

design a static state feedback control law $u = k(\x,v)$ such that the associated feedback system is linear.

\begin{nb}the system is linear w.r.t. input\end{nb}

\begin{example}
    Model of a centrifuge

    \[
        J \ddot\theta = -k \dot \theta^2 \sgn(\dot \theta) + u
    \]

    setting $x = \dot\theta$ the system can be written

    \[
    \begin{cases}
        \dot x = - \frac{k}{J} x^2 \sgn(x) + \frac{1}{J} u \\
        y = x
    \end{cases}
    \]

    if we set $v = - \frac{k}{J}x^2 \sgn(x) + \frac{1}{J}u$ the system become

    \[
    \begin{cases}
        \dot x = v \\
        y = x
    \end{cases}
    \]

    the feedback system seen from outside seems linear, and a controller for linear system can be made using the input $v$ and the output $y$.
\end{example}

\subsection{Relative degree}

Before to go on, let us introduce the concept of \textbf{relative degree} of a system.

Given a system with output $y=\vect c(\x)$ its relative degree is defined as the minimum amount time derivation of $y$ such that it is directly dependent by input $u$.

\[
    r = \min \left\{ \gamma : \pd{y^{(\gamma)}}{u} \neq 0 \right\}
\]

The concept of relative degree can be formulated also exploiting the \textbf{Lie derivative}\footnote{Defined as $\Lie{f} h(\x) = \pd{h}{\x} f(\x) = \sum_{i=1}^{n} f_i(\x) \pd{h(\x)}{x_i}$}

\[
    \dot y =
    %\frac{d}{dt} \vect c(\x) =
    \pd{\vect c}{\x} \dx =
    \pd{\vect c}{\x} \left(\vect a(\x) + \vect b(\x) u\right) =
    \Lie{\vect a} \vect c(\x) + u \Lie{\vect b} \vect c(\x)
\]

if $\Lie{\vect b} \vect c(\x) \neq 0$ so the relative degree of the system is equal to $1$ else it is greater than $1$ or not well-defined;
in this case we can iterate the derivation of the output in order to find the relative degree.

Since $\Lie{\vect b} \vect c(\x) = 0$ then $\dot y = \Lie{\vect a} \vect c(\x)$ and

\[
    \ddot y =
    %\frac{d}{dt} \left( \Lie{\vect a} \vect c(\x) \right) =
    \pd{\left( \Lie{\vect a} \vect c(\x) \right)}{x} \dx =
    \pd{\left( \Lie{\vect a} \vect c(\x) \right)}{x} \left(\vect a(\x) + \vect b(\x) u\right) =
    \Lie{\vect a}^2 \vect c(\x) + u \Lie{\vect b} \Lie{\vect a} \vect c(\x)
\]

now if $\Lie{\vect b} \Lie{\vect a} \vect c(\x) \neq 0$ then the relative degree is $2$ else we can go on with the iterations.

The procedure can be generalized

\begin{equation}\label{eqn:output-via-lie}
    y^{(k)} = \Lie{\vect a}^k \vect c(\x) + u \Lie{\vect b} \Lie{\vect a}^{k-1} \vect c(\x) \qquad \Longleftarrow \Lie{\vect b} \Lie{\vect a}^{h-1} \vect c(\x) = 0, h < k
\end{equation}

so the relative degree can be defined as

\[
    r = \min \left\{ \gamma : \Lie{\vect b} \Lie{\vect a}^{\gamma-1} \vect c(\x) \neq 0 \right\}
\]

\subsection{Normal canonical form}

Given a nonlinear, affine, time-invariant, SISO system that can be rewritten in the \textbf{normal canonical form}, in particular a system of degree $n$ and relative degree $r$ with input $u$ and state $\x$ can be rewritten in the form

\[
    \begin{cases}
        \dot{\tilde x}_1 = \tilde x_2 \\
        \dot{\tilde x}_2 = \tilde x_3 \\
        \vdots \\
        \dot{\tilde x}_r = \tilde\lambda(\vect{\tilde x}) + \tilde\mu (\vect{\tilde x}) u \\
        \left.\begin{matrix}\dot{\tilde x}_{r+1} = \tilde\eta_1(\vect{\tilde x}) \\
        \vdots \\
        \dot{\tilde x}_{n} = \tilde\eta_{n-r}(\vect{\tilde x}) \\
        \end{matrix} \qquad \right. \\
        y = \tilde x_1
    \end{cases}
\]

where the state $\vect{\tilde x}$ is a transformation of the state $\x$ of the original system under the transformation $\vect{\tilde x} = \varphi(\vect x)$.

\subsubsection{Linearization of normal canonical form}

If we impose

\[
    u = \frac{1}{\tilde\mu (\vect{\tilde x})} \left( v - \tilde\lambda(\vect{\tilde x}) \right)
\]

we get a linear I/O map

\[
    \begin{cases}
        \dot{\tilde x}_1 = \tilde x_2 \\
        \dot{\tilde x}_2 = \tilde x_3 \\
        \vdots \\
        \dot{\tilde x}_r = v \\
        \left.\begin{matrix}\dot{\tilde x}_{r+1} = \tilde\eta_1(\vect{\tilde x}) \\
        \vdots \\
        \dot{\tilde x}_{n} = \tilde\eta_{n-r}(\vect{\tilde x}) \\
        \end{matrix} \qquad \right\} \text{zero dynamics} \\
        y = \tilde x_1
    \end{cases}
\]

\begin{nb}if $r=n$ the "zero dynamics" part disappears, and we get a fully linearized system\end{nb}

\section{Linearization}

So the goal given a generic nonlinear, affine, time-invariant, SISO system is to define a transformation $\vect{\tilde x} = \varphi(\vect x)$ which defines an equivalent system in the \textbf{normal canonical form}.

From the definition of the \textbf{relative degree}(\cref{eqn:output-via-lie}) we have

\[
    y^{(k-1)}=\Lie{a}^{k-1} c(\x),\qquad k = 1,2,\dots, r-1
\]

so we can impose $\tilde x_k =\varphi_k(\x) = \Lie{a}^{k-1} c(\x)$, then

\begin{multline*}
    \dot{\tilde x}_k =
    \pd{\varphi_k}{\x} \dx =
    \pd{\left( \Lie{a}^{k-1} c(\x) \right)}{\x} \left( a(\x) + b(\x) u \right) \\ =
    \Lie{a}^k c(\x) + \cancelto{0}{u \Lie{b}\Lie{a}^{k-1} c(\x)} =
    \Lie{a}^k c(\x) =
    \varphi_{k+1}(\x) =
    \tilde x_{k+1}
\end{multline*}

so
\[
    \dot{\tilde x}_k = \tilde x_{k+1},\qquad k = 1,2,\dots, r-1
\]

instead the differential of $\dot{\tilde x}_r$ can be written as

\begin{equation}\label{eqn:newx-r}
    \dot{\tilde x}_r = \Lie{a}^r c(\x) + u \Lie{b}\Lie{a}^{r-1} c(\x)
\end{equation}

\begin{nb}the transformation $\tilde x_k =\varphi_k(\x)$ imposes $y^{(k-1)} = \tilde x_k,\quad k = 1,2,\dots,r$\end{nb}

\begin{theorem}[input-output state feedback linearization]
If a nonlinear, affine, time-invariant, SISO system has \textbf{relative degree} $r$ then, one can get a linear I/O map via state feedback with the control law
    \[
        u = \frac{1}{\Lie{\vect b} \Lie{\vect a}^{r-1} \vect c(\x)} \left( v - \Lie{\vect a}^{r} \vect c(\x) \right)
    \]
\end{theorem}

\begin{proof}
    Using \cref{eqn:newx-r} we can define the equation for the control law as

    \[
        v = \Lie{\vect a}^{r} \vect c(\x) + u \Lie{\vect b} \Lie{\vect a}^{r-1} \vect c(\x)
    \]

    and if we merge the two equations we get

    \[
        \dot{\tilde x}_r = v \quad \implies \quad y^{(r)} = v
    \]
\end{proof}

\subsection{Full feedback linearization}

\begin{theorem}\label{thm:full-feedback-linearization}
A \textbf{full feedback linearization} of a nonlinear, affine, time-invariant, SISO system is possible if and only if a regular function $c(\cdot)$ can be found such that system
\[
    \begin{cases}
        \dx = \vect a(\x) + \vect b(\x) u \\
        y = \vect c(\x)
    \end{cases}
\]
has a \textbf{relative degree} $r$ equal to the order $n$ of the linear system.
\end{theorem}

so we get the linearized system

\[
    \begin{cases}
        \dxe = \tilde A \xe + \tilde B v \\
        y = \tilde C \xe
    \end{cases} \quad
    \tilde A = \begin{bmatrix}
            0 & 1 & 0 & \dots & 0 \\
            0 & 0 & 1 & \dots & 0 \\
            \dots & \dots & \dots & \dots & \dots \\
            0 & 0 & 0 & \dots & 1 \\
            0 & 0 & 0 & \dots & 0 \\
    \end{bmatrix},
    \tilde B = \begin{bmatrix}
                   0 \\ 0 \\ \vdots \\ 0 \\ 1
    \end{bmatrix},
    \tilde C = \begin{bmatrix}
                   1 \\ 0 \\ \vdots \\ 0 \\ 0
    \end{bmatrix}^{\trans}
\]

\subsubsection{Existence of $c(\cdot)$}

In order to satisfy the \cref{thm:full-feedback-linearization} is useful to have a theorem to check if $c(\cdot)$ exists.

\begin{theorem}
    A function $c(\cdot)$ that satisfies the \cref{thm:full-feedback-linearization} exists if and only if there exists an open set $D, \x^0 \in D$, such that the following conditions hold:

    \begin{itemize}
        \item The vectors in the set $\{\vect b(\x), \ad{\vect a}^1 \vect b(\x), \dots, \ad{\vect a}^{n-1} \vect b(\x) \}$\footnote{this is another common way to indicate the \textbf{Lie bracket} such that $\ad{\vect a}^{1}\vect b(\x) = \Lieb{\vect a}{\vect b} = \Lie{\vect a}\vect b(\x) - \Lie{\vect b}\vect a(\x) = \pd{\vect b}{\x}\vect a(\x) - \pd{\vect a}{\x}\vect b(\x)$ and $\ad{\vect a}^{i}\vect b(\x) = \Lieb{\vect a}{\ad{\vect a}^{i-1}\vect b}$} are linearly independent for all $x \in D$
        \item The set of the vector fields $\{\vect b(\x), \ad{\vect a}^1 \vect b(\x), \dots, \ad{\vect a}^{n-2} \vect b(\x) \}$ is involutive in $D$
    \end{itemize}
\end{theorem}

\begin{definition}[Involutive]
Let $\vect f_i : D \to \Real^n$ with $D \subseteq \Real^n$ be a \textbf{regular vector field}, $i= 1,2,\dots, m$.
    The set $\{\vect f_1, \vect f_2, \dots, \vect f_m\}$, is involutive if
    \begin{multline*}
        \rank \begin{bmatrix} \vect f_1(\x) & \vect f_2(\x) & \dots & \vect f_m(\x) \end{bmatrix} = \\
        \rank \begin{bmatrix} \vect f_1(\x) & \vect f_2(\x) & \dots & \vect f_m(\x) & \Lieb{\vect f_i(\x)}{\vect f_j(\x)} \end{bmatrix}, \\
        \quad \forall \x \in D, \quad i,j \in \{1,2,\dots, m\}
    \end{multline*}
\end{definition}

\begin{nb}if all the vectors $\vect f_i(\x)$ are constant(independent of $\x$) the set of the second condition  is automatically \textbf{involutive} because $\Lieb{\vect f_i(\x)}{\vect f_j(\x)} = 0$\end{nb}

\subsection{Pole assignment}

After the linearization we can impose the dynamics of the system with the method designed for linear systems like \textbf{pole assignment}.

If we choose $v = - K \xe + w$ the system becomes

\[
    \begin{cases}
        \dxe = (\tilde A - \tilde B K) \xe + \tilde B w \\
        y = \tilde C \xe
    \end{cases}
\]

then the dynamics of the system is assigned by the eigenvalues of $(\tilde A - \tilde B K)$.

\begin{gather*}
    K = \begin{bmatrix}
    k_n \\ k_{n-1} \\ k_{n-2} \\ \vdots \\ k_1
    \end{bmatrix}^{\trans}
    \quad \implies \quad
    \tilde A - \tilde B K = \begin{bmatrix}
        0 & 1 & 0 & \dots & 0 \\
        0 & 0 & 1 & \dots & 0 \\
        \dots & \dots & \dots & \dots & \dots \\
        0 & 0 & 0 & \dots & 1 \\
        -k_n & -k_{n-1} & -k_{n-2} & \dots & -k_1 \\
    \end{bmatrix}
\end{gather*}

which eigenvalues are the roots of the characteristic polynomial

\[
    \chi(s) = s^n + k_1 s^{n-1} + k_2 s^{n-2} + k_3 s^{n-3} + \dots + k_{n-1} s + k_n
\]


\printindex

\end{document}