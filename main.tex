\documentclass[12pt]{report}

\usepackage[utf8]{inputenc}
\usepackage{amssymb}
\usepackage{amsmath}
\usepackage{svg}
\usepackage{tabularx}
\usepackage{soul}
\usepackage[pdfusetitle]{hyperref}
\usepackage{enumitem}
\usepackage{nicefrac}

\usepackage{geometry}
\geometry{
a4paper,
left=30mm,
right=30mm,
top=40mm,
bottom=30mm
}

\setcounter{tocdepth}{3}
\setcounter{secnumdepth}{3}

\usepackage{tikz}
\usetikzlibrary{circuits}
\usetikzlibrary{shapes,arrows}
\usetikzlibrary{positioning}

\tikzset{auto,>=latex', minimum size=0pt, node distance=1, font=\fontsize{11}{11}\selectfont}

\tikzstyle{block} = [draw, fill=white, rectangle, minimum height=3em, minimum width=4em, text centered, align=center]
\tikzstyle{integrator} = [draw, fill=white, rectangle, minimum height=3em, minimum width=2em, font={$\frac{1}{s}$}]
\tikzstyle{sum} = [draw, fill=white, circle, node distance=1cm]
\tikzstyle{input} = [coordinate]
\tikzstyle{output} = [coordinate]
\tikzstyle{pinstyle} = [pin edge={to-,thin,black}]
\tikzstyle{gain} = [draw, isosceles triangle, minimum height = 3em, isosceles triangle apex angle=60]
\tikzstyle{spy} = [coordinate, inner sep=0, outer sep=0, minimum size=0]
\tikzstyle{node} = [draw, circle, fill, minimum size=2pt, inner sep=0pt, outer sep=0pt, anchor=center]

%\usepackage{bodegraph}

%\usetikzlibrary{intersections}
%\usetikzlibrary{calc}

\setlength{\parindent}{0em}
\setlength{\parskip}{0.5em}

\newenvironment{conditions}
{\par\vspace{\abovedisplayskip}\noindent\centering\begin{tabular}{>{$}r<{$} @{${}:{}$} l}}
{\end{tabular}\par\vspace{\belowdisplayskip}}

\newenvironment{nb}{\par\vspace{\abovedisplayskip}\noindent\begin{em}n.b.\ \ignorespaces}{\end{em}}


\DeclareMathOperator{\im}{im}
\DeclareMathOperator{\rad}{rad}
\DeclareMathOperator{\sgn}{sgn}
\DeclareMathOperator{\diag}{diag}

\newcommand{\matr}[1]{\boldsymbol{#1}}
\newcommand{\vect}[1]{\boldsymbol{#1}}

\newcommand{\trans}{\mathsf{T}}
\newcommand{\inv}{{-1}}
\newcommand{\psinv}{\dag}
\newcommand\norm[1]{\left\lVert#1\right\rVert}

\newcommand{\Real}{\mathbb{R}}

\newcommand{\I}{\matr{I}}
\newcommand{\0}{\vect{0}}

\newcommand{\J}{\matr{J}}
\newcommand{\dJ}{\dot{\J}}

\newcommand{\estimate}[1]{\hat{#1}}
\newcommand{\error}[1]{\tilde{#1}}
\newcommand{\reference}[1]{\bar{#1}}

\newcommand{\B}{\matr{B}}
\newcommand{\dB}{\dot\B}
\newcommand{\Bs}{\hat\B}
\newcommand{\Be}{\error\B}

\newcommand{\C}{\matr{C}}
\newcommand{\Cs}{\estimate\C}
\newcommand{\Ce}{\error\C}

\newcommand{\g}{\vect{g}}
\newcommand{\gs}{\estimate\g}

\newcommand{\n}{\vect{n}}
\newcommand{\ns}{\estimate\n}
\newcommand{\ner}{\error\n} % #r becuase ne already exists

\newcommand{\K}{\matr{K}}

\newcommand{\y}{\vect{y}}
\newcommand{\dy}{\dot{\y}}
\newcommand{\ddy}{\ddot{\y}}

\newcommand{\V}{V}
\newcommand{\dV}{\dot{\V}}

\newcommand{\q}{\vect{q}}
\newcommand{\dq}{\dot{\q}}
\newcommand{\ddq}{\ddot{\q}}
\newcommand{\qd}{\bar{\q}}
\newcommand{\dqd}{\dot{\reference{\q}}}
\newcommand{\ddqd}{\ddot{\reference{\q}}}
\newcommand{\qe}{\error{\q}}
\newcommand{\dqe}{\dot{\error{\q}}}
\newcommand{\ddqe}{\ddot{\error{\q}}}
\newcommand{\qs}{\estimate{\q}}
\newcommand{\dqs}{\dot{\estimate{\q}}}
\newcommand{\ddqs}{\ddot{\estimate{\q}}}

\newcommand{\x}{x}
\newcommand{\dx}{\dot{\x}}
\newcommand{\ddx}{\ddot{\x}}
\newcommand{\xd}{\reference{\x}}
\newcommand{\dxd}{\dot{\reference{\x}}}
\newcommand{\ddxd}{\ddot{\reference{\x}}}
\newcommand{\xe}{\error{\x}}
\newcommand{\dxe}{\dot{\error{\x}}}
\newcommand{\ddxe}{\ddot{\error{\x}}}
\newcommand{\xs}{\estimate{\x}}
\newcommand{\dxs}{\dot{\estimate{\x}}}
\newcommand{\ddxs}{\ddot{\estimate{\x}}}

\newcommand{\warning}[1]{\textcolor{red}{WARNING #1}}

\usepackage{imakeidx}
\makeindex[intoc]

\usepackage[backend=biber]{biblatex}
\addbibresource{references.bib}

\title{Nonlinear control\\Notes\thanks{the whole document is based on \cite{cir-slides}}}
\author{Roberto Bochet}

\begin{document}

\maketitle

\printbibliography

\tableofcontents

\chapter{Passive systems}

\section{Definition}

\begin{gather*}
    \begin{cases}
        \dx (t) = f(\x,u) \\
        y(t) = g(\x,u)
    \end{cases} \\
    f(\vect 0,0) = \vect 0, \quad g(\vect 0,0)=0
\end{gather*}

A dynamical system is called \textbf{passive} if there exists a \textbf{Lyapunov function} $\V(\x)$ called \textbf{storage function}, such that

\[
    uy \geq \pd{V}{\x}(\x) f(\x,u) + \epsilon u^2 + \delta y^2 + \rho \psi(\x)
\]

with

\[
    \epsilon \geq 0, \delta \geq 0, \rho \geq 0, \psi(\x) > 0
\]

Notice that the Lyapunov function can be written as

\[
     \pd{\V}{\x} f(\x,0) = \dV(\x)
\]

\subsection{Conservative system}

A passive system is called \textbf{conservative} if $\epsilon = \delta = \rho = 0$, such that

\[
    uy = \pd{V}{\x}(\x) f(\x,u)
\]

\subsection{Strictly passive}

A passive system is called strictly passive if any of $\epsilon, \delta, \rho$ is greater that $0$.
In particular

Strictly passive w.r.t the input if $\epsilon > 0$ \\
Strictly passive w.r.t the output if $\delta > 0$ \\
Strictly passive w.r.t the state if $\rho > 0$

\subsection{Zero-state observable}

A system is \textbf{zero-state observable} if $\x(t) = 0$ is the only \textbf{free evolution} (u=0) of the state compatible with identically zero output (y=0)

\section{Properties}

\begin{theorem}
    If a system is \textbf{passive} with positive definite storage function, then $\x=0$ is a \textbf{stable} equilibrium for the system with zero input
\end{theorem}

\begin{theorem}
    If a system is \textbf{strictly passive w.r.t the state} with positive definite storage function, then $\x=0$ is an \textbf{asymptotically stable} equilibrium for the system with zero input
\end{theorem}

\begin{theorem}
    If a system is \textbf{strictly passive w.r.t the output} and \textbf{zero-stable observable} with positive definite storage function, then $\x=0$ is an \textbf{asymptotically stable} equilibrium for the system with zero input
\end{theorem}

\begin{theorem}
    If the storage function associated to \textbf{asymptotically stable} equilibrium is radially unbounded then the equilibrium is \textbf{GAS}
\end{theorem}

\begin{theorem}
    If a system is \textbf{zero state observable} and \textbf{passive} with positive definite storage function and \textbf{radially unbounded}, then $\x=0$ is a \textbf{GAS equilibrium} of the output feedback system with control law
    \[
        u=-\phi(y),\qquad \phi(0)=0, \quad y\phi(y)>0, \forall y \neq 0
    \]
\end{theorem}

\printindex

\end{document}