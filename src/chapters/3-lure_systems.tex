\chapter{Lur'e systems}

\section{Definition}

\begin{figure}[htb]
    \centering
    \resizebox{0.7\textwidth}{!}{
        \begin{tikzpicture}
            \node [sum] (sum_e) {};

            \node [block, right=1cm of sum_e] (controller) {$\varphi(\cdot)$};
            \node [block, right=1cm of controller] (system) {$G(s)$};

            \node [output, right=1cm of system] (oy) {};

            \draw [->] (sum_e) -- node {$e$} (controller);
            \draw [->] (controller) -- node {$u$} (system);
            \draw [->] (system) -- node [pos=0.8] {$y$} node [node, name=y, pos=0.5] {} (oy);

            \draw [->] (y) -- ++(0,-1.5) -| node [pos=0.95] {$-$} (sum_e);
        \end{tikzpicture}
    }
    \caption{Autonomous Lur'e system}
    \label{fig:lure-system}
\end{figure}

\textbf{Autonomous Lur'e system} is composed by a \textbf{linear system} ($G(s)$) \textbf{controllable} and \textbf{observable} fed by a \textbf{nonlinear function} ($\varphi(e)$) on the feedback error.

\[
    \begin{cases}
        \dx = A \x + B \varphi(-C\x) \\
        y = C \x
    \end{cases}
\]

The non linear function is bounded into a sector ($\varphi(e) \in \Phi_{[k_1,k_2]}$) defined as

\[
    \Phi_{[k_1,k_2]} = \{\phi(\cdot) : (k_2 e -u)(u - k_1 e) \geq 0, u=\phi(e), \forall e \in \Real \}
\]

For this kind of system $\bar{x} = 0$ is an equilibrium.

\subsection{Meaning}

\begin{figure}[htb]
    \centering
    \resizebox{.9\textwidth}{!}{
    \begin{tikzpicture}
        \node [input] (iy_d) {};

        \node [sum, right=1cm of iy_d] (sum_e) {};
        \node [block, right=0.6cm of sum_e] (controller) {$R(s)$};
        \node [sum, right=0.6cm of controller] (sum_u) {};
        \node [block, right=0.6cm of sum_u] (nonlinearity) {$\Psi(\cdot)$};
        \node [block, right=1cm of nonlinearity] (system) {$P(s)$};
        \node [sum, right=0.6cm of system] (sum_d) {};

        \node [output, right=1cm of sum_d] (oy) {};

        \node [input, above=1cm of sum_u] (iu_n) {};
        \node [input, above=1cm of sum_d] (id_n) {};

        \node [block, below=1cm of nonlinearity] (feedback) {$T(s)$};

        \draw [->] (iy_d) -- node{$y^0$} (sum_e);
        \draw [->] (sum_e) -- node {$e$} (controller);
        \draw [->] (controller) -- node {$w$} (sum_u);
        \draw [->] (sum_u) -- node {$u$} (nonlinearity);
        \draw [->] (nonlinearity) -- node {$m$} (system);
        \draw [->] (system) -- (sum_d);

        \draw [->] (sum_d) -- node [pos=0.8] {$\tilde y$} node [node, name=y, pos=0.5] {} (oy);

        \draw [->] (y) |- (feedback);
        \draw [->] (feedback) -| node[pos=0.07] {$y$} node [pos=0.95] {$-$} (sum_e);

        \draw [->] (iu_n) -- node[pos=0.2] {$u^0$} (sum_u);
        \draw [->] (id_n) -- node[pos=0.2] {$d^0$} (sum_d);

    \end{tikzpicture}
    }
    \caption{A generic feedback system with a nonlinearity}
    \label{fig:generic-feedback-system}
\end{figure}

A general feedback system (e.g \cref{fig:generic-feedback-system}) containing a nonlinear function where input are constant can be traced to an \textbf{autonomous Lur'e system}.

Chosen an equilibrium (associated to a constant inputs) and the whole system can be studied in the neighbourhood of it applying a coordinate shifting (i.e. $\Delta x(t) = x(t) - \bar x$)

From the definition of constant inputs we get $\Delta y^0 = \Delta u^0 = \Delta d^0 = 0$, then we get a system like \cref{fig:generic-feedback-system-equilibrium}, easily to retract back to an \textbf{Autonomous Lur'e system}

\begin{figure}[htb]
    \centering
    \resizebox{.8\textwidth}{!}{
        \begin{tikzpicture}
            \node [sum] (sum_e) {};
            \node [block, right=0.6cm of sum_e] (controller) {$R(s)$};
            \node [block, right=2cm of controller] (nonlinearity) {$\varphi(\cdot)$};
            \node [block, right=1cm of nonlinearity] (system) {$P(s)$};

            \node [output, right=1cm of system] (oy) {};

            \node [block, below=1cm of nonlinearity] (feedback) {$T(s)$};

            \draw [->] (sum_e) -- node {$\Delta e$} (controller);
            \draw [->] (controller) -- node {$\Delta w = \Delta u$} (nonlinearity);
            \draw [->] (nonlinearity) -- node {$\Delta m$} (system);
            \draw [->] (system) -- node [pos=0.8] {$\Delta \tilde y$} node [node, name=y, pos=0.5] {} (oy);

            \draw [->] (y) |- (feedback);
            \draw [->] (feedback) -| node[pos=0.07] {$\Delta y$} node [pos=0.95] {$-$} (sum_e);


        \end{tikzpicture}
    }
    \caption{A generic feedback system with a nonlinearity considered near the equilibrium}
    \label{fig:generic-feedback-system-equilibrium}
\end{figure}

\section{Properties}

\begin{definition}
    A \textbf{autonomous Lur'e system} is \textbf{absolutely stable in the sector $[k_1,k_2]$} if $\x=\vect 0$ is a $GAS$ equilibrium, for every sector nonlinearity $\varphi(\cdot) \in \Phi_{[k_1,k_2]}$
\end{definition}

\begin{theorem}[Nyquist criterion]\label{thm:nyquist}
A degenerated \textbf{autonomous Lur'e system} with $\varphi(e) = k e$ (linear in the error) is asymptotically stable if and only if the Nyquist plot of the linear part encircles (anti-clockwise) the point of the complex plane $-\frac{1}{k}$ as many times as the amount the unstable poles of it
\end{theorem}

\begin{theorem}[Necessary condiction]\label{thm:lure-nyquist}
    \[
        I(k_1,k_2) = \{\alpha \in \Real : \alpha = -\frac{1}{k}, k \in [k_1,k_2]\}
    \]
    If \textbf{autonomous Lur'e system} is absolutely stable in the sector $[k_1,k_2]$, then the \textbf{Nyquist plot} of the linear part encircles (anti-clockwise) $I(k_1,k_2)$ as many times as the amount the unstable poles of it
\end{theorem}

\begin{nb}\cref{thm:lure-nyquist} defined a necessary condition for stability of a \textbf{autonomous Lur'e system} but not sufficient\end{nb}

\begin{conjecture}[Aizerman conjecture, 1949]\label{cjt:aizerman}
    The \cref{thm:lure-nyquist} is also sufficient for the stability of the \textbf{autonomous Lur'e system}
\end{conjecture}

\begin{nb}
    it is possible find a counterexample for the \cref{cjt:aizerman}
\end{nb}

\subsection{Popov criterion}

\begin{theorem}[Popov criterion, 1962]\label{thm:popov-criterion}
    An \textbf{autonomous Lur'e system} is absolutely stable in sector $[0,k]$ if the linear part is \textbf{asymptotically stable} and exists a real number $q$ such that
    \[
        \Re [(1+\j \omega q) G(\j \omega)] > - \frac{1}{k}, \forall \omega \geq 0
    \]
\end{theorem}

\begin{nb}\cref{thm:popov-criterion} defined a sufficient but not necessary condition for the stability of the \textbf{autonomous Lur'e system}\end{nb}

\begin{theorem}[Popov criterion (alternative), 1962]
    An \textbf{autonomous Lur'e system} is absolutely stable in sector $[0,k]$ if the linear part is \textbf{asymptotically stable} with transfer function $G(s)$ and exists a positive real number $q$ such that $-\frac{1}{q}$ is not an eigenvalue of the matrix A of the linear part and defined
    \[
        H(s) = 1+k(1+qs)G(s)
    \]
    $H(s)$ is strictly positive real
\end{theorem}

\subsubsection{Special case $q=0$}

With $q=0$ the condition is reduced to

\[
    \Re [G(\j \omega)] > - \frac{1}{k}, \forall \omega \geq 0
\]

so the \cref{thm:popov-criterion} requires that the $G(\j\omega)$ polar plot is contained in the right-half-plane defined by the vertical line passing to the real point $-\frac{1}{k}$

\subsubsection{Case $q \neq 0$}

\begin{corollary}
    Defining
    \[
        G^*(\j\omega) = \Re[G(\j\omega)] + \j\omega \Im[G(\j\omega)]
    \]
    the \cref{thm:popov-criterion} requires that exists a straight line passing through the real point $-\frac{1}{k}$ such that the $G^*(\j\omega)$ polar plot is contained in the right-half-plane defined by it
\end{corollary}

\subsubsection{Popov criterion and Aizenman conjecture}

Let us denote with $K_P$ the largest $K$ value such that the absolute stability of an \textbf{autonomous Lur'e system} in the sector $[0,K]$ is guaranteed via \textbf{Popov criterion};
and with $K_N$ the largest $K$ such that the degenerated \textbf{autonomous Lur'e system} ($\varphi(\cdot)$ linearly in error) is asymptotically stable via \cref{thm:nyquist}(\textbf{Nyquist criterion})

\begin{nb} for an \textbf{autonomous Lur'e system} $K_P \leq K_N$\end{nb}

\begin{nb}if $K_P = K_N$ then, the system satisfies the \textbf{Aizenman Conjecture} \end{nb}

\begin{nb}$K_P < K_N$ does not imply that the system does not satisfy the \textbf{Aizenman Conjecture}\end{nb}

\subsection{Generalized Popov criterion for sector $[k_1,k_2]$}

Consider an \textbf{autonomous Lur'e system} with $\varphi(\cdot) \in \Phi_{[k_1,k_2]}$, then we add and subtract $k_1 e$ from $u$, like in \cref{fig:lure-system-reworked}, notice that the new system is completely equivalent to the original one

\begin{figure}[htb]
    \centering
    \resizebox{0.7\textwidth}{!}{
        \begin{tikzpicture}[node distance = 1cm, auto]
            \tikzset{node distance = 1cm and 2cm}
            \node [sum] (sum_e) {};
            \node [node, right=.6cm of sum_e] (e_2) {};
            \node [block, right=.6cm of e_2] (controller) {$\varphi(\cdot)$};
            \node [block, below=.6cm of controller] (k1_1) {$k_1$};
            \node [sum, right=1cm of controller] (sum_v) {};
            \node [sum, right=1cm of sum_v] (sum_u) {};
            \node [block, right=1cm of sum_u] (system) {$G(s)$};
            \node [block, below=.6cm of system] (k1_2) {$k_1$};
            \node [node, right=.6cm of system] (y_1) {};
            \node [node, right=.6cm of y_1] (y) {};

            \node [output, right=.6cm of y] (oy) {};

            \draw [->] (sum_e) -- node {$e$} (controller);
            \draw [->] (e_2) |- (k1_1);

            \draw [->] (controller) -- (sum_v);
            \draw [->] (k1_1) -| node [pos=0.9] {$-$} (sum_v);
            \draw [->] (sum_v) -- node {$v$} (sum_u);
            \draw [->] (sum_u) -- node {$u$} (system);
            \draw [->] (system) -- node [pos=0.8] {$y$} (oy);
            \draw [->] (y_1) |- (k1_2);
            \draw [->] (k1_2) -| node [pos=0.9] {$-$} (sum_u);

            \draw [->] (y) -- ++(0,-3) -| node [pos=0.95] {$-$} (sum_e);
        \end{tikzpicture}
    }
    \caption{Autonomous Lur'e system reworked}
    \label{fig:lure-system-reworked}
\end{figure}

Now, we rewrite the block diagram in the standard shape of a \textbf{autonomous Lur'e system} like in the \cref{fig:lure-system-reqworked-compact}, where

\begin{gather*}
    F(s) = \frac{G(s)}{1+k_1 G(s)} \\
    \eta(e) = \varphi(e) - k_1 e, \qquad \eta(e) \in \Phi_{[0,k_2-k_1]}
\end{gather*}

\begin{figure}[htb]
    \centering
    \resizebox{0.7\textwidth}{!}{
        \begin{tikzpicture}
            \node [sum] (sum_e) {};

            \node [block, right=1cm of sum_e] (controller) {$\eta(\cdot)$};
            \node [block, right=1cm of controller] (system) {$F(s)$};

            \node [output, right=1cm of system] (oy) {};

            \draw [->] (sum_e) -- node {$e$} (controller);
            \draw [->] (controller) -- node {$u$} (system);
            \draw [->] (system) -- node [pos=0.8] {$y$} node [node, name=y, pos=0.5] {} (oy);

            \draw [->] (y) -- ++(0,-1.5) -| node [pos=0.95] {$-$} (sum_e);
        \end{tikzpicture}
    }
    \caption{Compact autonomous Lur'e system reworked}
    \label{fig:lure-system-reqworked-compact}
\end{figure}

So, due to the equivalence of the original system and the reworked one we can formulate a generalized version of the \textbf{Popov criterion}

\begin{theorem}\label{thm:popov-criterion-generalized}
    An \textbf{autonomous Lur'e system} is absolutely stable in sector $[k_1,k_2]$ if $F(s)$ is asymptotically stable and exists a real number $q$ such that
\[
    \Re \left[(1+\j \omega q) F(s)\right] > - \frac{1}{k_2-k_1}, \forall \omega \geq 0
\]
is satisfied when
\[
    F(s) = \frac{G(s)}{1+k_1 G(s)}
\]
\end{theorem}

\subsubsection{The special case $q=0$}

The condition required by the \cref{thm:popov-criterion-generalized} with an imposed $q=0$ became $F(s)$ asymptotically stable and
\[
    \Re \left[F(s)\right] > - \frac{1}{k_2-k_1}, \; \forall \omega \geq 0
\]

The first condition can be checked remembering that $F(s)$ is a feedback system composed by the loop function $k_1 G(s)$, so exploiting the \textbf{Nyquist criterion} we can assert that $F(s)$ is an asymptotically stable system if the \textbf{Nyquist plot} of $G(s)$ encircles (anti-clockwise) the point $-\frac{1}{k_1}$ as many times the number of unstable poles of $G(s)$.

The second one can be rewrite like a condition on the polar plot of $G(s)$ instead of the one of $F(s)$. Considering that
\[
    G(s) = \frac{F(s)}{1 - k_1 F(s)}
\]

we can map the right-half-plane ($>-\frac{1}{k_2 - k_1}$) in which the polar plot of $F(s)$ must be contained, to a circle where the polar plot of $G(s)$ must not go in. This circle is defined as

\[
    O(k_1,k_2) = \{\tilde{s} \in \Complex : \tilde{s} = \frac{s}{1-k_1 s}, s = \alpha + \j \beta, \alpha \leq - \frac{1}{k_2 - k_1}\}
\]

Merging the two conditions we can formulate the \textbf{Circle criterion}

\begin{theorem}[Circle criterion]
    An \textbf{autonomous Lur'e system} is absolutely stable in sector $[k_1,k_2]$ if the \textbf{Nyquist plot} of $G(s)$ encircles (anti-clockwise) the circle $O(k_1, k_2)$ as many time the number of unstable poles of $G(s)$, with
    \[
        O(k_1,k_2) = \{\tilde{s} \in \Complex : \tilde{s} = \frac{s}{1-k_1 s}, s = \alpha + \j \beta, \alpha \leq - \frac{1}{k_2 - k_1}\}
    \]
\end{theorem}

\begin{corollary}\label{crl:cicle-criterion}
    If the sector $[k_1,k_2]$ includes zero ($k_1 < 0 < k_2$) and system $G(s)$ is asymptotically stable, then the \textbf{autonomous Lur'e system} is absolutely stable in sector $[k_1,k_2]$ if the \textbf{Nyquist plot} of $G(s)$ \textbf{is contained} in the circumference passing in $-\frac{1}{k_1}$ and $-\frac{1}{k_2}$ with the center lies on the real axis.
\end{corollary}

\begin{nb}The \cref{crl:cicle-criterion} is easily verified computing $O(k_1,k_2)$ for $k_1 < 0 < k_2$\end{nb}

\subsection{Time-varying system}

\begin{figure}[htb]
    \centering
    \resizebox{0.7\textwidth}{!}{
        \begin{tikzpicture}
            \node [sum] (sum_e) {};

            \node [block, right=1cm of sum_e] (controller) {$\varphi(\cdot,t)$};
            \node [block, right=1cm of controller] (system) {$G(s)$};

            \node [output, right=1cm of system] (oy) {};

            \draw [->] (sum_e) -- node {$e$} (controller);
            \draw [->] (controller) -- node {$u$} (system);
            \draw [->] (system) -- node [pos=0.8] {$y$} node [node, name=y, pos=0.5] {} (oy);

            \draw [->] (y) -- ++(0,-1.5) -| node [pos=0.95] {$-$} (sum_e);
        \end{tikzpicture}
    }
    \caption{Time-varying autonomous Lur'e system}
    \label{fig:lure-system-time-varying}
\end{figure}

Some conclusion on the stability of a \textbf{time-varying system} can be derived exploiting the autonomous Lur'e system theory.

Let us take into account a \textbf{Time-varying autonomous Lur'e system} (\cref{fig:lure-system-time-varying}) with

\[
    \varphi(\cdot,t) \in \Phi_{[k_1,k_2]} = \{\phi(\cdot) : k_1 e^2 \geq \phi(e)e \geq k_2 e^2, \forall e \in \Real \}, \quad \forall t \in \Real
\]

The system can be written as

\[
    \begin{cases}
        \dx = A \x + B \varphi(-C\x, t) \\
        y = C \x
    \end{cases}
\]

Let us consider the \textbf{Time-varying autonomous Lur'e system} with $\varphi$ linear in the error(\cref{fig:lure-system-time-varying-linear}), such that

\[
    \varphi(e,t) = k(t)e \in \Phi_{[k_1,k_2]}, \forall t
\]

thus $k_1 \leq k(t) \leq k_2$

\begin{figure}[htb]
    \centering
    \resizebox{0.7\textwidth}{!}{
        \begin{tikzpicture}
            \node [sum] (sum_e) {};

            \node [block, right=1cm of sum_e] (controller) {$k(t)$};
            \node [block, right=1cm of controller] (system) {$G(s)$};

            \node [output, right=1cm of system] (oy) {};

            \draw [->] (sum_e) -- node {$e$} (controller);
            \draw [->] (controller) -- node {$u$} (system);
            \draw [->] (system) -- node [pos=0.8] {$y$} node [node, name=y, pos=0.5] {} (oy);

            \draw [->] (y) -- ++(0,-1.5) -| node [pos=0.95] {$-$} (sum_e);
        \end{tikzpicture}
    }
    \caption{Time-varying autonomous Lur'e system linear in error}
    \label{fig:lure-system-time-varying-linear}
\end{figure}

\begin{theorem}
    Time-varying autonomous Lur'e system(\cref{fig:lure-system-time-varying}) is \textbf{absolutely stable in $[k_1,k_2]$ sector}
    if and only if the equivalent Time-varying autonomous Lur'e system linear in the error(\cref{fig:lure-system-time-varying-linear})
    is \textbf{GUAS} $\forall k(\cdot) : k(t) \in [k_1,k_2], \forall t$
\end{theorem}

\subsubsection{Stability of time-varying system linear in error}

The system in \cref{fig:lure-system-time-varying-linear} can be write as

\[
    \begin{cases}
        \dx = A \x - B k(t) C\x = (A - k(t) B C) \x = A_{k(t)} \x \\
        y = C \x
    \end{cases}
\]

with this formulation theory on switching system(\cref{subsec:switching-stability-arbitrary}) can be used to prove the stability.

Based on \cref{thm:switching-stability-linear-common} we can state that

\begin{theorem}
    Time-varying autonomous Lur'e system linear in the error (\cref{fig:lure-system-time-varying-linear}) with $k(t) \in [k_1,k_2], \forall t$
    is \textbf{GAUS} if there exists  $P=P^\trans \succ 0$ such that
    \[
        A_{k_2}^\trans P + P A_{k_2} \prec 0 \text{\quad and \quad}
        A_{k_1}^\trans P + P A_{k_1} \prec 0
    \]
\end{theorem}

\begin{corollary}
    Time-varying autonomous Lur'e system linear in the error (\cref{fig:lure-system-time-varying-linear}) with $k(t) \in [k_1,k_2], \forall t$ is \textbf{GAUS} stability is \textbf{exponential}
\end{corollary}
