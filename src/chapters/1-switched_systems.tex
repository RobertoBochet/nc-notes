\chapter{Switched systems}

\section{Definition}

A switched system is a system whom the state dynamics is described by a set of \textbf{more than one} vector fields;
at specific moment the state dynamics is controlled by only one of these vector fields.
The switching between these dynamics can be controlled by exogenous variable (\textbf{time-dependent switching}) or defined in function of the state (\textbf{state-dependent switching}).

The system is defined by

\[
	\begin{cases}
		\dx = \vect f_q(\x), \qquad q \in Q = \{1,2,\dots,m\} \\
		\sigma(\cdot): U \to Q
	\end{cases}
\]

a family of $m$ systems characterized by their vector fields $\vect f_q(\x)$ and a function $\sigma(\cdot)$ which determines the current one is in use.

\begin{figure}[htb]
	\centering
	\resizebox{0.7\textwidth}{!}{
		\begin{tikzpicture}
			\node [node] (switch) {};

			\node [node, above left=1 and 1 of switch] (switch_1) {};
			\node [block, left=1 of switch_1] (f_1) {$\vect f_1(\x)$};
			\draw (f_1) -- (switch_1);

			\node [node, below left=1 and 1 of switch] (switch_m) {};
			\node [block, left=1 of switch_m] (f_m) {$\vect f_m(\x)$};
			\draw (f_m) -- (switch_m);

			\node [node,left=0.5 of f_m] (x_f_m) {};
			\draw [->] (x_f_m) -- (f_m);

			\draw [shorten <=5] (switch_1) -- (switch);

			\draw [->, bend right, looseness=1, shorten <=5, shorten >=5] (switch_1) to node{$\sigma$} (switch_m);

			\node [integrator, right=1 of switch] (int) {};

			\node [output, right=1 of int] (ox) {};

			\draw [->] (switch) -- node {$\dx$} (int);
			\draw [->] (int) -- node [pos=0.8] {$x$} node [node, name=x, pos=0.5] {} (ox);

			\draw [->] (x) -- ++(0,-2) -| (x_f_m) |- (f_1);

			\draw [draw=none] (f_1) to node[node, pos=0.25] {} node[node, pos=0.5] {} node[node, pos=0.75] {} (f_m);
		\end{tikzpicture}
	}
	\caption{Switched system}
\end{figure}

\section{Equilibrium}

\begin{theorem}
	If a switched system defined by
	\[
		\dx = \vect f_q(\x), \qquad q \in Q = \{1,2,\dots,m\}
	\]
	has $\vect f_q(\vect 0) = \vect 0, \forall q \in Q$ then $\x = \vect 0$ is an equilibrium of the switched system.
\end{theorem}

\begin{nb}the stability of an equilibrium of the $m$ systems of the family tells us nothing about the stability of switched system\end{nb}

\section{Stability}

\subsection{Stability for arbitrary switching}

\begin{definition}[globally uniformly asymptotically stable equilibrium]
	The equilibrium $\x=\vect 0$ is \textbf{GUAS} if it is \textbf{GAS}, uniformly w.r.t. the switching signals $\sigma$
\end{definition}

\begin{theorem}
	If the family of systems share a \textbf{radially unbounded common Lyapunov function} then the equilibrium $\x=\vect 0$ of the switched system is \textbf{GUAS}.
\end{theorem}

\subsubsection{Linear system}

A switched system can have a family systems of linear ones such that

\[
	\dx = A_q \x, \qquad q \in Q = \{1,2,\dots,m\}
\]

\begin{theorem}
	If for a switched system with a family systems of linear ones exists a matrix $P = P^\trans > 0$ such that
	\[
		PA_q + A_q^\trans P < 0, \qquad \forall q \in Q = \{1,2,\dots,m\}
	\]
	then, the equilibrium $\x=\vect 0$ of the switched system is \textbf{GUAS}.
\end{theorem}

\begin{theorem}
	If a switched system with a family systems of linear ones has $A_q, \forall q \in Q = \{1,2,\dots,m\}$ \textbf{Hurwitz}\footnote{All of its eigenvalues have negative real part} and \textbf{commutative}\footnote{$A_i A_j = A_j A_i, \forall i,j \in Q = \{1,2,\dots,m\}$} then, the equilibrium $\x=\vect 0$ of the switched system is \textbf{GUAS}.
\end{theorem}

\begin{theorem}
	If a switched system with a family systems of linear ones has $A_q, \forall q \in Q = \{1,2,\dots,m\}$ \textbf{Hurwitz} and \textbf{triangular} then, the equilibrium $\x=\vect 0$ of the switched system is \textbf{GUAS}.
\end{theorem}

\subsection{Stability for slow switching}

\begin{theorem}
	Given a switched system with a family systems of linear ones where $A_q, \forall q \in Q = \{1,2,\dots,m\}$ is \textbf{Hurwitz} then exists a \textbf{dwell time} $\tau_D \leq t_{q,i}$, (where $t_{q,i}$ is the duration of the i-th signal $q$ given by $\sigma(\cdot)$) such that the equilibrium $\x=\vect 0$ of the switched system is \textbf{GAS}.
\end{theorem}

\subsection{Stability under state-dependent switching}

\begin{definition}[state-dependent common Lyapunov function]
	A switched systems has a \textbf{state-dependent common Lyapunov function} at $\x=\vect 0$ if there exists a $C^1$ function $V: \Real^n \to \Real$ is such that $V(\x) > 0, \forall \x \neq \vect 0, V(\vect 0) = 0$
	\[
		\pdiff{V}{\x}(\x) f_{\sigma(\x)}(\x) < 0, \forall \x \in \Real^n
	\]
	where $\sigma : X \to Q$
\end{definition}

\begin{theorem}
	Given a switched systems it there exists a \textbf{radially unbounded state-dependent common Lyapunov function} then, the equilibrium $\x=\vect 0$ of the switched system is \textbf{GAS}.
\end{theorem}

\begin{nb}this condition is less stringent than requiring a \textbf{shared radially unbounded common Lyapunov function}\end{nb}

\subsubsection{Stability under state-dependent switching for linear systems}

\begin{theorem}
	Given a switched system with a family systems of linear ones where $A_q, \forall q \in Q = \{1,2,\dots,m\}$ not necessarily \textbf{Hurwitz} if there exists a
	\[
		A = \sum_{i=1}^m \alpha_i A_i, \qquad \alpha_i \geq 0, \sum_{i=1}^m \alpha_i =1
	\]
	such than $A$ is \textbf{Hurwitz} then there exists a \textbf{state-dependent switching strategy} such that the equilibrium $\x=\vect 0$ of the switching system is \textbf{GAS}.
\end{theorem}