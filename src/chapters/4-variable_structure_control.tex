\chapter{Variable structure control}

\section{Definition}

\begin{figure}[htb]
    \centering
    \resizebox{0.7\textwidth}{!}{
        \begin{tikzpicture}
            \node [sum] (sum_e) {};

            \node [block, right=1cm of sum_e] (controller) {$\psi(\cdot)$};
            \node [block, above=1cm of controller] (switching_law) {$s(\cdot) > 0$};
            \node [block, right=1cm of controller] (system) {$G(s)$};

            \node [output, right=1cm of system] (oy) {};

            \draw [->] (sum_e) -- node {$e$} (controller);
            \draw [->] (switching_law) -- (controller);
            \draw [->] (controller) -- node {$u$} (system);
            \draw [->] (system) |- node [pos=0.8] {$x$} (switching_law);
            \draw [->] (system) -- node [pos=0.8] {$y$} node [node, name=y, pos=0.5] {} (oy);

            \draw [->] (y) -- ++(0,-1.5) -| node [pos=0.95] {$-$} (sum_e);
        \end{tikzpicture}
    }
    \caption{Example of VSC}
    \label{fig:vsc-example}
\end{figure}

In \cref{fig:vsc-example} is shown an example of the structure of a typical \textbf{VSC}, in particular we have a piecewise control law, based on the switching function $s(x) > 0$.

\[
    \psi(e) = \begin{cases}
                  \varphi_1(e), \qquad s(x) > 0 \\
                  \varphi_2(e), \qquad s(x) < 0
    \end{cases}
\]

given this control law we get two different overall systems, one based on the control law for $s(x) > 0$ and one for $s(x) < 0$ (\cref{fig:vcs-two-systems}).

\begin{figure}[htb]
    \centering
    \begin{subfigure}[b]{0.45\textwidth}
        \resizebox{\textwidth}{!}{
            \begin{tikzpicture}
                \node [sum] (sum_e) {};

                \node [block, right=1cm of sum_e] (controller) {$\varphi_1(\cdot)$};
                \node [block, right=1cm of controller] (system) {$G(s)$};

                \node [output, right=1cm of system] (oy) {};

                \draw [->] (sum_e) -- node {$e$} (controller);
                \draw [->] (controller) -- node {$u$} (system);
                \draw [->] (system) -- node [pos=0.8] {$y$} node [node, name=y, pos=0.5] {} (oy);

                \draw [->] (y) -- ++(0,-1.5) -| node [pos=0.95] {$-$} (sum_e);
            \end{tikzpicture}
        }
        \caption{$s(x)>0$}
        \label{fig:vcs-two-systems-a}
    \end{subfigure}
    \hfill
    \begin{subfigure}[b]{0.45\textwidth}
        \resizebox{\textwidth}{!}{
            \begin{tikzpicture}
                \node [sum] (sum_e) {};

                \node [block, right=1cm of sum_e] (controller) {$\varphi_2(\cdot)$};
                \node [block, right=1cm of controller] (system) {$G(s)$};

                \node [output, right=1cm of system] (oy) {};

                \draw [->] (sum_e) -- node {$e$} (controller);
                \draw [->] (controller) -- node {$u$} (system);
                \draw [->] (system) -- node [pos=0.8] {$y$} node [node, name=y, pos=0.5] {} (oy);

                \draw [->] (y) -- ++(0,-1.5) -| node [pos=0.95] {$-$} (sum_e);
            \end{tikzpicture}
        }
        \caption{$s(x)<0$}
        \label{fig:vcs-two-systems-b}
    \end{subfigure}
    \caption{The two systems under the switching law}
    \label{fig:vcs-two-systems}
\end{figure}

\begin{nb}the stability of the two systems tells us nothing about the stability of the overall one\end{nb}

Considering the vector fields drive the evolution of the states of the two systems (\cref{fig:vcs-two-systems}), the switching law defines a new piecewise vector field merging these two.

The inside boundaries of the new piecewise vector field can be of two kinds:

\begin{itemize}
    \item Across switching surface

    the state reaches the boundary while following some dynamics, crosses it, and continues its evolution according to the other dynamics

    \item Attractive switching surface

    the state reaches the boundary and cannot leave it because the vector fields on both sides are pointing towards the boundary;
    the state can evolve only sliding along the boundary, the system lose a dof (sliding mode)
\end{itemize}

\section{Design}

Given a linear time-invariant SISO system of order $n$ with $(A,B)$ controllable and $(A,C)$ observable, we want design a variable structure controller such that $y(t)$ tends to some (constant) reference signal $\bar y^0$ in some reasonable amount of time, for all $\bar y^0$ and for all possible initial state.

Suppose to put in the controllable canonical form the linear system, so the system can be defined as

\[
    \begin{cases}
        \dx_i = x_{i+1}, \qquad i = 1,2,\dots, n-1 \\
        \dx_n = - a_n x_1 - a_{n-1} x_2 - \dots - a_1 x_n + u \\
        y = c_n x_1 + c_{n-1} x_2 + \dots + c_1 x_n
    \end{cases}
\]

or in the compact form

\begin{gather*}
    \begin{cases}
        \dx = A \x + B u \\
        y = C \x
    \end{cases} \\
    A = \begin{bmatrix}
            0 & 1 & 0 & \dots & 0 & 0 \\
            0 & 0 & 1 & \dots & 0 & 0 \\
            0 & 0 & 0 & \dots & 0 & 0 \\
            \dots & \dots & \dots & \dots & \dots & \dots \\
            0 & 0 & 0 & \dots & 0 & 1 \\
            -a_n & -a_{n-1} & -a_{n-2} & \dots & -a_2 & -a_1
    \end{bmatrix},
    B = \begin{bmatrix} 0 \\ 0 \\ 0 \\ \vdots \\ 0 \\ 1 \end{bmatrix},
    C^{\trans} = \begin{bmatrix} c_n \\ c_{n-1} \\ c_{n-2} \\ \vdots \\ c_2 \\ c_1 \end{bmatrix}
\end{gather*}

\subsection{Design the switching function}

A possible choice for the switching function can be

\[
    s(\vect x) = \beta_{n-1} x_1 + \beta_{n-2} x_2 + \dots + \beta_1 x_{n-1} + x_n - \bar w
\]

on the boundary $s(\vect x)=0$ the system is subjected to the constraint

\[
    x_n = - \beta_{n-1} x_1 - \beta_{n-2} x_2 - \dots - \beta_1 x_{n-1} + \bar w
\]

so, the overall system became

\[
    \begin{cases}
        \dx_i = x_{i+1}, \qquad i = 1,2,\dots, n-2 \\
        \dx_{n-1} = x_n = - \beta_{n-1} x_1 - \beta_{n-2} x_2 - \dots - \beta_1 x_{n-1} + \bar w \\
        y = c_n x_1 + c_{n-1} x_2 + \dots + c_1 x_n
    \end{cases}
\]

this system has the characteristic polynomial defined as

\[
    \chi(s) = s^{n-1} + \beta_1 s^{n-2} + \dots + \beta_{n-1}
\]

which roots can be arbitrarily assigned such that the system became asymptotically stable with a single equilibrium defined as

\[
    \begin{cases}
        \bar x_i = 0, \qquad i = 2,3,\dots, n-1 \\
        \bar x_1 = \frac{\bar w}{\beta_{n-1}} \\
        \bar y = c_n \bar x_1
    \end{cases}
\]

and imposing $\bar w = \frac{\beta_{n-1}}{c_n} \bar y^0$ we get $\bar y = \bar y^0$ and the switching law become

\begin{equation}
    s(\vect x) = \vect \beta^\trans \vect x - \gamma \bar y^0, \qquad
    \vect \beta^\trans = \begin{bmatrix} \beta_{n-1} & \beta_{n-2} & \dots & \beta_1 & 1 \end{bmatrix},
    \gamma = \frac{\beta_{n-1}}{c_n}
    \label{eqn:switching-law}
\end{equation}

So, the switching law drives the state from any point of $s(\vect x)=0$ to the equilibrium $\bar y = \bar y^0$ following an arbitrary dynamic.

\subsection{Design the control law}

The switching law will drive the state from any point $s(\vect x) = 0$ to the arbitrarily equilibrium, so the control law $u(\vect x)$ will have to drive the state from any point of the vector space to a point of the boundary $s(\vect x) = 0$ in finite time.

The goal: impose the dynamics of $s(\vect x)$ such that from any point of vector field the state reaches $s(\vect x) = 0$.

We shall adopt the so-called "reaching-law approach" to impose the reaching condition.

Defining the dynamics of the switching law such that Lyapunov function

\[
    V(s) = \frac{1}{2} s^2
\]

has negative time derivative satisfying

\[
    \diff{V}{t} = s \dot s \leq - \eta |s|, \quad \eta > 0
\]

Let us propose the dynamics

\[
    \dot s = -q \sgn(s) - r g(s), \qquad q> 0, r \geq 0, g(\cdot) : g(-s)=-g(s), sg(s) > 0, \forall s \neq 0
\]

for which

\[
    \diff{V}{t} = s \dot s = -q \sgn(s) s - r s g(s) = -q |s| - r s g(s) \leq - q |s|
\]

that satisfy the reaching condition in finite time (with $\eta = q$).

\subsubsection{Impose the switching law dynamics}

Differencing with respect the time the \cref{eqn:switching-law} we get

\[
    \dot s = \vect \beta^\trans \dot{\vect x} = \vect \beta^\trans (A \vect x + B u)
\]

\begin{align*}
    \vect \beta^\trans (A \vect x + B u) &= -q \sgn(s) - r g(s) \\
    u &= - \vect \beta^\trans A \vect x -q \sgn(s) - r g(s) \\
    u &= - \vect \beta^\trans A \vect x -q \sgn(\vect \beta^\trans \vect x - \gamma \bar y^0) - r g(\vect \beta^\trans \vect x - \gamma \bar y^0) \\
    u &= - \vect \beta^\trans A \vect x + q \sgn(\gamma \bar y^0 - \vect \beta^\trans \vect x) - r g(\vect \beta^\trans \vect x - \gamma \bar y^0) \\
    \intertext{$g(s)$ is a odd function from definition, $g(-s)=-g(s)$}
    u &= - \vect \beta^\trans A \vect x + q \sgn(\gamma \bar y^0 - \vect \beta^\trans \vect x) + r g(\gamma \bar y^0 - \vect \beta^\trans \vect x)
\end{align*}

The resulting control system is shown in \cref{fig:vsc-control}.

\begin{nb}in order to avoid infinitely fast switching a good rule can be to implement the $\sgn(\cdot)$ function as a 2-level hysteresis switching\end{nb}

\begin{figure}[htb]
    \centering
    \resizebox{\textwidth}{!}{
        \begin{tikzpicture}
            \node [input] (iy_d) {};

            \node [block, right=0.6cm of iy_d] (gamma) {$\gamma$};
            \node [sum, right=.6cm of gamma] (sum_e) {};
            \node [block, right=1cm of sum_e] (sgn) {$\sgn(\cdot)$};
            \node [block, above=.5cm of sgn] (g) {$g(\cdot)$};
            \node [block, right=0.6cm of sgn] (q) {$q$};
            \node [block, above=.5cm of q] (r) {$r$};

            \node [block, below=2cm of sgn] (beta) {$\vect \beta^\trans$};

            \node [sum, right=0.6cm of q] (sum_u) {};
            \node [block, right=1cm of sum_u] (system) {$G(s)$};
            \node [block, below=.8cm of system] (alpha) {$\vect \beta^\trans A$};

            \node [output, right=1cm of system, yshift=0.3cm] (oy) {};

            \node [node, right=.5 of alpha] (xalpha) {};

            \draw [->] (iy_d) -- node[pos=0.2] {$\bar y^0$} (gamma);

            \draw [->] (gamma) -- (sum_e);
            \draw [->] (sum_e) -- node[node, name=s2, pos=.5] {} node[swap]{$-s$} (sgn);
            \draw [->] (s2) |- (g);
            \draw [->] (sgn) -- (q);
            \draw [->] (g) -- (r);
            \draw [->] (q) -- (sum_u);
            \draw [->] (r) -| (sum_u);
            \draw [->] (sum_u) -- node {$u$} (system);

            \draw [->] (alpha) -| node[pos=0.9] {$-$} (sum_u);

            \draw [->] (system)[yshift=-.3cm] -| node[pos=0.6] {$x$} (xalpha) |- (beta);
            \draw [->] (xalpha) -- (alpha);

            \draw [->] (beta) -| node[pos=0.95] {$-$} node[pos=0.7] {$\tilde y$} (sum_e);

            \draw [->] (system.east |- oy) -- node [pos=0.7] {$y$} (oy);
        \end{tikzpicture}
    }
    \caption{Designed control law}
    \label{fig:vsc-control}
\end{figure}

The so designed control law guaranteed a transitory time $\leq \frac{|s(\x_0)|}{q}$

Good performance can be reached also with $r = 0$ because the transitory time is bound to $q$, but a good choice of $r$ and $g(s)$ can enhance the performance while it is not affecting the excursion of $u$.

A valid choice of $g(s)$ can be $g(s) = s$ and $r=1$

\section{Variations}

\subsection{Robustness w.r.t. load disturbance}

\begin{figure}[htb]
    \centering
    \resizebox{\textwidth}{!}{
        \begin{tikzpicture}
            \node [input] (iy_d) {};

            \node [block, right=0.6cm of iy_d] (gamma) {$\gamma$};
            \node [sum, right=.6cm of gamma] (sum_e) {};
            \node [block, right=1cm of sum_e] (sgn) {$\sgn(\cdot)$};
            \node [block, above=.5cm of sgn] (g) {$g(\cdot)$};
            \node [block, right=0.6cm of sgn] (q) {$q$};
            \node [block, above=.5cm of q] (r) {$r$};

            \node [block, below=2cm of sgn] (beta) {$\vect \beta^\trans$};

            \node [sum, right=0.6cm of q] (sum_v) {};
            \node [sum, right=0.6cm of sum_v] (sum_u) {};
            \node [block, right=1cm of sum_u] (system) {$G(s)$};
            \node [block, below=.8cm of system] (alpha) {$\vect \beta^\trans A$};

            \node [output, right=1cm of system, yshift=0.3cm] (oy) {};

            \node [node, right=.5 of alpha] (xalpha) {};

            \draw [->] (iy_d) -- node[pos=0.2] {$\bar y^0$} (gamma);

            \draw [->] (gamma) -- (sum_e);
            \draw [->] (sum_e) -- node[node, name=s2, pos=.5] {} node[swap]{$-s$} (sgn);
            \draw [->] (s2) |- (g);
            \draw [->] (sgn) -- (q);
            \draw [->] (g) -- (r);
            \draw [->] (q) -- (sum_v);
            \draw [->] (r) -| (sum_v);
            \draw [->] (sum_v) -- node {$v$} (sum_u);
            \draw [->] (sum_u) -- node {$u$} (system);

            \draw [->] (alpha) -| node[pos=0.9] {$-$} (sum_u);

            \draw [->] (system)[yshift=-.3cm] -| node[pos=0.6] {$x$} (xalpha) |- (beta);
            \draw [->] (xalpha) -- (alpha);

            \draw [->] (beta) -| node[pos=0.95] {$-$} node[pos=0.7] {$\tilde y$} (sum_e);

            \draw [->] (system.east |- oy) -- node [pos=0.7] {$y$} (oy);

            \draw [->] (sum_u) ++ (0,1) -- node[pos=0.1] {$w$} (sum_u);
        \end{tikzpicture}
    }
    \caption{VSC with disturbance on the control variable}
    \label{fig:vsc-control-noise}
\end{figure}

Let us consider a VSC with noise on the control variable as in \cref{fig:vsc-control-noise}, the overall system can write as

\[
    \begin{cases}
        \dx = A\x + B u \\
        \tilde y = \vect\beta^\trans \x
    \end{cases}
\]

where $u = - \vect\beta^\trans A \x + v + w$, so

\[
    \begin{cases}
        \dx = (A - B\vect\beta^\trans A)\x + B (v + w) \\
        \tilde y = \vect\beta^\trans \x
    \end{cases}
\]

if we calculate the first derivation of $\tilde y$ we get

\[
    \dot{\tilde y} = \vect\beta^\trans \dx =  (\vect\beta^\trans A - \vect\beta^\trans B\vect\beta^\trans A)\x + \vect\beta^\trans B (v + w)
\]

and remembering that due to the shape $\vect\beta^\trans B = 1$ we get

\[
    \dot{\tilde y} = v + w = q \sgn(\gamma\bar y^0 - \tilde y) + r g(\gamma\bar y^0 - \tilde y) + w
\]

with $\abs{w(t)} = W < q$ the equilibrium still defined by $\tilde y=\gamma \bar y^0$ (because the sign of $\dot{\tilde y}$ is defined by the function $\sgn(\gamma \bar y^0 - \tilde y)$).

Now let us check the transitory time, setting $V(s) = \frac{1}{2}s^2$ where $s = \tilde y - \gamma \bar y^0$

\begin{multline*}
    \dV = s\dot s = s \dot{\tilde y}
    = - s q \sgn(s) - r g(s) + s w \\
    = - s q \sgn(s) - r g(s) + \abs{s} \sgn(s) w \\
    = - \abs{s} (q - w \sgn(s)) - r g(s)
\end{multline*}

due to $r g(s) > 0$ from definition we get a bound on $\dV$

\[
    \dV \leq -(q-W) \abs{s}
\]

Recalling the upper bound law $t_r \leq \frac{\abs{s(x(0))}}{\eta}$ valid if $\dV \leq - \eta \abs{s}$ we get

\[
    t_r \leq \frac{\abs{s(x(0))}}{q-W}
\]

\begin{theorem}
    If a VCS is affected by a noise $w(t)$ on the control variable $u$ such that $\abs{w(t)} = W < q$, then its equilibrium is not influenced by the noise but the reaching time bound become
    \[
        t_r \leq \frac{\abs{s(x(0))}}{q-W}
    \]
\end{theorem}

\subsection{Non-directly accessible state}

Under the assumption $(A,C)$ is observable, although the state is not directly measurable, we can design a \textbf{asymptotic state observer} in order to get an estimation of the system's state, then we can use it in the control law instead of the real one.

The procedure for design the switching function and control law will be the same for the available state case, it will be added only an \textbf{asymptotic state observer} fed by the \textbf{control law} and the \textbf{system's output} as in \cref{fig:vsc-control-nostate}.

The control law become

\[
    u = - \vect \beta^\trans A \xs + q \sgn(\gamma \bar y^0 - \vect \beta^\trans \xs) + r g(\gamma \bar y^0 - \vect \beta^\trans \xs)
\]

with

\[
    \dxs = A\xs + Bu + L(y - C\xs)
\]

$L$ can be arbitrarily chosen(due to $(A,C)$ is observable) in order to assign the wanted dynamic to the prediction error

\[
    \de = (A - LC) \e, \qquad \e = \x - \xs
\]

so

\[
    \norm{\e(t)} \leq \mu e^{-\lambda_0 t}\norm{e(0)}, t \geq 0
\]

where $\lambda_0$ is the the real part of the slowest pole of the error dynamics.

\begin{nb}after the transitory ends during which $e\ne 0$, the system with observer reaches exactly the same performance of the equivalent system without the observer because $\x = \xs$\end{nb}

\begin{figure}[htb]
    \centering
    \resizebox{\textwidth}{!}{
        \begin{tikzpicture}
            \node [input] (iy_d) {};

            \node [block, right=0.6cm of iy_d] (gamma) {$\gamma$};
            \node [sum, right=.6cm of gamma] (sum_e) {};
            \node [block, right=1cm of sum_e] (sgn) {$\sgn(\cdot)$};
            \node [block, above=1cm of sgn] (g) {$g(\cdot)$};
            \node [block, right=0.6cm of sgn] (q) {$q$};
            \node [block, above=1cm of q] (r) {$r$};

            \node [block, below=2cm of sgn] (beta) {$\vect \beta^\trans$};

            \node [sum, right=0.6cm of q] (sum_u) {};
            \node [block, right=1cm of sum_u] (system) {$G(s)$};
            \node [block, below=.8cm of sum_u] (alpha) {$\vect \beta^\trans A$};
            \node [block] at (beta -| system) (observer) {$O$};

            \node [output, right=1cm of system] (oy) {};
            \node [node] at (observer -| alpha) (x) {};

            %\node [node, right=.5 of alpha] (xalpha) {};

            \draw [->] (iy_d) -- node[pos=0.2] {$\bar y^0$} (gamma);

            \draw [->] (gamma) -- (sum_e);
            \draw [->] (sum_e) -- node[node, name=s2, pos=.5] {} node[swap]{$-s$} (sgn);
            \draw [->] (s2) |- (g);
            \draw [->] (sgn) -- (q);
            \draw [->] (g) -- (r);
            \draw [->] (q) -- (sum_u);
            \draw [->] (r) -| (sum_u);
            \draw [->] (sum_u) -- node[node, name=u] {} node {$u$} (system);

            \draw [->] (alpha) -- node[pos=0.7] {$-$} (sum_u);

            %\draw [->] (system)[yshift=-.3cm] -| node[pos=0.6] {$x$} (xalpha) |- (beta);
            \draw [->] (x) -- (alpha);

            \draw [->] (beta) -| node[pos=0.9] {$-$} (sum_e);

            \draw [->] (system.east |- oy) -- node[node, name=y] {} node [pos=0.7] {$y$} (oy);

            \draw [->] (y) |- (observer);

            \draw [->] (observer) -- node [pos=0.3] {$\xs$} (beta);

            \draw [->] (u) -- ++(0,-0.5) -- (observer.north);
        \end{tikzpicture}
    }
    \caption{System with state not accessible}
    \label{fig:vsc-control-nostate}
\end{figure}

\subsection{Avoid high frequency switching of the control input}

Introducing a $\sgn$ function implemented exploiting a hysteresis cycle avoids the \textbf{infinite switching problem} of the control variable, but the \textbf{high frequency switching} of it still.

\begin{nb}the $\sgn$ also implemented with hysteresis cycle has a output like a square wave, such that the gotten control variable value is composed also by high frequency components\end{nb}

A possible solution to avoid this issue could be to introduce an additional state to the system in order to filtrate the control variable value like an integration on it

\[
    u = \frac{1}{s} v
\]

\begin{figure}[htb]
    \centering
    \resizebox{\textwidth}{!}{
        \begin{tikzpicture}
            \node [input] (iy_d) {};

            \node [block, right=0.6cm of iy_d] (gamma) {$\gamma$};
            \node [sum, right=.6cm of gamma] (sum_e) {};
            \node [block, right=1cm of sum_e] (sgn) {$\sgn(\cdot)$};
            \node [block, above=1cm of sgn] (g) {$g(\cdot)$};
            \node [block, right=0.6cm of sgn] (q) {$q$};
            \node [block, above=1cm of q] (r) {$r$};

            \node [block, below=2cm of sgn] (beta) {$\tilde{\vect \beta}^\trans$};

            \node [sum, right=0.6cm of q] (sum_u) {};
            \node [integrator, right=0.6cm of sum_u] (int) {};
            \node [block, right=0.6cm of int] (system) {$G(s)$};
            \node [block, below=.8cm of system] (alpha) {$\tilde{\vect \beta}^\trans \tilde A$};

            \node [output, right=1cm of system, yshift=0.3cm] (oy) {};

            \node [node, right=.5 of alpha] (xalpha) {};

            \draw [->] (iy_d) -- node[pos=0.2] {$\bar y^0$} (gamma);

            \draw [->] (gamma) -- (sum_e);
            \draw [->] (sum_e) -- node[node, name=s2, pos=.5] {} node[swap]{$-s$} (sgn);
            \draw [->] (s2) |- (g);
            \draw [->] (sgn) -- (q);
            \draw [->] (g) -- (r);
            \draw [->] (q) -- (sum_u);
            \draw [->] (r) -| (sum_u);
            \draw [->] (sum_u) -- node {$v$} (int);
            \draw [->] (int) -- node {$u$} (system);

            \draw [->] (alpha) -| node[pos=0.9] {$-$} (sum_u);

            \draw [->] (system)[yshift=-.3cm] -| node[pos=0.6] {$x$} (xalpha) |- (beta);
            \draw [->] (xalpha) -- (alpha);

            \draw [->] (beta) -| node[pos=0.9] {$-$} (sum_e);

            \draw [->] (system.east |- oy) -- node [pos=0.7] {$y$} (oy);
        \end{tikzpicture}
    }
    \caption{Designed controller with integrator on control law}
    \label{fig:vsc-control-int}
\end{figure}